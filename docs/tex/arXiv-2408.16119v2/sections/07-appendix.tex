\section{Supplementary Materials}

Please check out our supplementary materials for (1) \df introduction video,  (2) unedited videos demonstrating \df's experiences to complete tasks in \autoref{sec:illustartive-scenarios} and user study tasks~\autoref{sec:evaluation}, and (3) user study handout.

\begin{comment}
\section{System Implementation Details}

\bpstart{Currently supported chart types}  \df  supports 16  charts types across five categories (scatter, bar, line statistical, custom). We listed out the Vega-Lite mark(s) used to compose the chart as well as visual channels provided to the user. 

\medskip

\begin{center}
\small
    \begin{tabular}{|c|c|l|}
       \hline Chart Type  & Vega-Lite Mark(s) & Visual channels \\\hline
        Scatter Plot & circle & \textsf{x}, \textsf{y}, \textsf{color}, \textsf{column}, \textsf{row} \\
        Ranged Dot Plot & circle + line & \textsf{x}, \textsf{y}, \textsf{color}\\ \hdashline
        
        Bar Chart & bar & \textsf{x}, \textsf{y}, \textsf{color}, \textsf{column}, \textsf{row} \\
        Grouped Bar Chart & bar & \textsf{x}, \textsf{y}, \textsf{group} \\
        Stacked Bar Chart & bar & \textsf{x}, \textsf{y}, \textsf{color}, \textsf{column}, \textsf{row} \\
        \hdashline
        Line Chart & line & \textsf{x}, \textsf{y}, \textsf{color}, \textsf{column}, \textsf{row} \\
        Dotted Line Chart & circle + line & \textsf{x}, \textsf{y}, \textsf{color}, \textsf{column}, \textsf{row} \\
        Heat Map & rect & \textsf{x}, \textsf{y}, \textsf{color}, \textsf{column}, \textsf{row} \\
        \hdashline
        Linear Regression & circle + line & \textsf{x}, \textsf{y}, \textsf{color}, \textsf{column}, \textsf{row}\\
        Histogram & bar & \textsf{x}, \textsf{color}, \textsf{column}, \textsf{row} \\
        {Boxplot} & boxplot &  \textsf{x}, \textsf{y}, \textsf{color}, \textsf{opacity}, \textsf{column}, \textsf{row} \\
        \hdashline
        Custom Point & circle & \textsf{x}, \textsf{y}, \textsf{color}, \textsf{size}, \textsf{shape}, \textsf{opacity}, \textsf{column}, \textsf{row} \\
        Custom Line & line & \textsf{x}, \textsf{y}, \textsf{detail}, \textsf{color}, \textsf{opacity}, \textsf{column}, \textsf{row} \\
        Custom Bar & bar & \textsf{x}, \textsf{y}, \textsf{color}, \textsf{size}, \textsf{shape}, \textsf{opacity}, \textsf{column}, \textsf{row} \\
        Custom Rect & rect & \textsf{x}, \textsf{y}, \textsf{x2}, \textsf{y2}, \textsf{color}, \textsf{size}, \textsf{opacity}, \textsf{column}, \textsf{row} \\
         Custom Area & area & \textsf{x}, \textsf{y}, \textsf{x2}, \textsf{y2}, \textsf{color}, \textsf{opacity}, \textsf{column}, \textsf{row} \\\hline
    \end{tabular}
\end{center}

\medskip

\df includes chart types with overlapping expressiveness (e.g., ``stacked bar chart'' can also be composed from ``custom bar'') so that novice users won't be overwhelmed with unfamiliar visual channels when working with basic charts. \df includes a glyph icon for each chart to assist user navigation. If a developer would like to extend \df to support new chart types, it can be achieved easily by providing a Vega-Lite template along with rules about how visual encodings from the user will be routed to different slots in the template. 

\bpstart{The system prompt} We provide the system prompt used by \df to communicate with the LLM. The system prompt includes role definition and few-shot examples. New user inputs are appended as \code{[CONTEXT]} and \code{[GOAL]} to end the system prompt, and the LLM is asked to complete the \code{[OUTPUT]} section.

\definecolor{LightGray}{gray}{0.95}
\begin{minted}[
baselinestretch=1.2,
breakanywhere,
bgcolor=LightGray,
fontsize=\footnotesize,
]{text}
You are a data scientist to help user to transform data that will be used for visualization.
The user will provide you information about what data would be needed, and your job is to create a python function based on the input data summary, transformation instruction and expected fields.
The users' instruction includes "expected fields" that the user want for visualization, and natural langauge instructions "goal" that describe what data is needed.

Concretely, you should first refine users' goal and then create a python function in the [OUTPUT] section based off the [CONTEXT] and [GOAL]:

    1. First, refine users' [GOAL]. The main objective in this step is to check if "visualization_fields" provided by the user are sufficient to achieve their "goal". Concretely:
        (1) based on the user's "goal", elaborate the goal into a "detailed_instruction".
        (2) determine "output_fields", the desired fields that the output data should have to achieve the user's goal, it's a good idea to include intermediate fields here.
        (2) now, determine whether the user has provided sufficient fields in "visualization_fields" that are needed to achieve their goal:
            - if the user's "visualization_fields" are sufficient, simply copy it.
            - if the user didn't provide sufficient fields in "visualization_fields", add missing fields in "visualization_fields" (ordered them based on whether the field will be used in x,y axes or legends);
                - "visualization_fields" should only include fields that will be visualized (do not include other intermediate fields from "output_fields")  
                - when adding new fields to "visualization_fields", be efficient and add only a minimal number of fields that are needed to achive the user's goal. generally, the total number of fields in "visualization_fields" should be no more than 3 for x,y,legend.

    Prepare the result in the following json format:

```
{
    "detailed_instruction": "..." // string, elaborate user instruction with details if the user
    "output_fields": [...] // string[], describe the desired output fields that the output data should have based on the user's goal, it's a good idea to preserve intermediate fields here (i.e., the goal of transformed data)
    "visualization_fields": [] // string[]: a subset of fields from "output_fields" that will be visualized, ordered based on if the field will be used in x,y axes or legends, do not include other intermediate fields from "output_fields".
    "reason": "..." // string, explain why this refinement is made
}
```

    2. Then, write a python function based on the refined goal, the function input is a dataframe "df" and the output is the transformed dataframe "transformed_df". "transformed_df" should contain all "output_fields" from the refined goal.
The python function must follow the template provided in [TEMPLATE], do not import any other libraries or modify function name. The function should be as simple as possible and easily readable.
If there is no data transformation needed based on "output_fields", the transformation function can simply "return df".

[TEMPLATE]

```python
import pandas as pd
import collections
import numpy as np

def transform_data(df):
    # complete the template here
    return transformed_df
```

    3. The [OUTPUT] must only contain a json object representing the refined goal (including "detailed_instruction", "output_fields", "visualization_fields" and "reason") and a python code block representing the transformation code, do not add any extra text explanation.

For example:

[CONTEXT]

Here are our datasets, here are their field summaries and samples:

table_0 (us_covid_cases) fields:
	Date -- type: object, values: 1/1/2021, 1/1/2022, 1/1/2023, ..., 9/8/2022, 9/9/2020, 9/9/2021, 9/9/2022
	Cases -- type: int64, values: -23999, -14195, -6940, ..., 1018935, 1032159, 1178403, 1433977

table_0 (us_covid_cases) sample:
```
|Date|Cases
0|1/21/2020|1
1|1/22/2020|0
2|1/23/2020|0
3|1/24/2020|1
4|1/25/2020|1
......
```

[GOAL]

{
    "instruction": "calculate 7-day moving average",
    "visualization_fields": ["Date", "7-day average cases"]
}

[OUTPUT]

{  
    "detailed_instruction": "Calculate the 7-day moving average of COVID-19 cases over time.",  
    "output_fields": ["Date", "Cases", "7-day average cases"],  
    "visualization_fields": ["Date", "7-day average cases"],  
    "reason": "To calculate the 7-day moving average, the 'Cases' field is required, but it is not needed for visualization. The provided fields are sufficient to achieve the goal."  
}  

```python
import pandas as pd  
import collections  
import numpy as np  
  
def transform_data(df):  
    # Convert Date column to datetime  
    df['Date'] = pd.to_datetime(df['Date'])  
      
    # Sort the dataframe by Date  
    df = df.sort_values('Date')  
      
    # Calculate the 7-day moving average of cases  
    df['7-day average cases'] = df['Cases'].rolling(window=7).mean()  
      
    # Select the output fields  
    transformed_df = df[['Date', 'Cases', '7-day average cases']]  
      
    return transformed_df  
```
\end{minted}


% [CONTEXT]

% Here are our datasets, here are their field summaries and samples:

% table_0 (weather_seattle_atlanta) fields:
% 	Date -- type: object, values: 1/1/2020, 1/10/2020, 1/11/2020, ..., 9/6/2020, 9/7/2020, 9/8/2020, 9/9/2020
% 	City -- type: object, values: Atlanta, Seattle
% 	Temperature -- type: int64, values: 30, 31, 32, ..., 83, 84, 85, 86

% table_0 (weather_seattle_atlanta) sample:
% ```
% |Date|City|Temperature
% 0|1/1/2020|Seattle|51
% 1|1/1/2020|Atlanta|45
% 2|1/2/2020|Seattle|45
% 3|1/2/2020|Atlanta|47
% 4|1/3/2020|Seattle|48
% ......
% ```

% [GOAL]

% {
%     "instruction": "create a scatter plot to with seattle and atlanta temperatures on x,y axes, color points by which city is warmer",
%     "visualization_fields": []
% }

% [OUTPUT]

% {  
%     "detailed_instruction": "Create a scatter plot to compare Seattle and Atlanta temperatures with Seattle temperatures on the x-axis and Atlanta temperatures on the y-axis. Color the points by which city is warmer.",  
%     "output_fields": ["Date", "Seattle Temperature", "Atlanta Temperature", "Warmer City"],  
%     "visualization_fields": ["Seattle Temperature", "Atlanta Temperature", "Warmer City"],  
%     "reason": "To compare Seattle and Atlanta temperatures with Seattle temperatures on the x-axis and Atlanta temperatures on the y-axis, and color points by which city is warmer, separate temperature fields for Seattle and Atlanta are required. Additionally, a new field 'Warmer City' is needed to indicate which city is warmer."  
% }  

% ```python
% import pandas as pd  
% import collections  
% import numpy as np  
  
% def transform_data(df):  
%     # Pivot the dataframe to have separate columns for Seattle and Atlanta temperatures  
%     df_pivot = df.pivot(index='Date', columns='City', values='Temperature').reset_index()  
%     df_pivot.columns = ['Date', 'Atlanta Temperature', 'Seattle Temperature']  
      
%     # Determine which city is warmer for each date  
%     df_pivot['Warmer City'] = df_pivot.apply(lambda row: 'Atlanta' if row['Atlanta Temperature'] > row['Seattle Temperature'] else 'Seattle', axis=1)  
      
%     # Select the output fields  
%     transformed_df = df_pivot[['Date', 'Seattle Temperature', 'Atlanta Temperature', 'Warmer City']]  
      
%     return transformed_df 
%```

\end{comment}