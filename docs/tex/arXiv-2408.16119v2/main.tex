%%
%% This is file `sample-sigconf.tex',
%% generated with the docstrip utility.
%%
%% The original source files were:
%%
%% samples.dtx  (with options: `all,proceedings,bibtex,sigconf')
%% 
%% IMPORTANT NOTICE:
%% 
%% For the copyright see the source file.
%% 
%% Any modified versions of this file must be renamed
%% with new filenames distinct from sample-sigconf.tex.
%% 
%% For distribution of the original source see the terms
%% for copying and modification in the file samples.dtx.
%% 
%% This generated file may be distributed as long as the
%% original source files, as listed above, are part of the
%% same distribution. (The sources need not necessarily be
%% in the same archive or directory.)
%%
%%
%% Commands for TeXCount
%TC:macro \cite [option:text,text]
%TC:macro \citep [option:text,text]
%TC:macro \citet [option:text,text]
%TC:envir table 0 1
%TC:envir table* 0 1
%TC:envir tabular [ignore] word
%TC:envir displaymath 0 word
%TC:envir math 0 word
%TC:envir comment 0 0
%%
%%
%% The first command in your LaTeX source must be the \documentclass
%% command.
%%
%% For submission and review of your manuscript please change the
%% command to \documentclass[manuscript, screen, review]{acmart}.
%%
%% When submitting camera ready ② to TAPS, please change the command
%% to \documentclass[sigconf]{acmart} or whichever template is required
%% for your publication.
%%
%%
%\documentclass[acmlarge, authorversion=true, nonacm=true]{acmart}
\documentclass[sigconf, authorversion=true]{acmart}

\newcommand{\tool}{Data Formulator 2\xspace}
\newcommand{\df}{\textsc{Df2}\xspace}
%\newcommand{\tool}{Data Anvil\xspace}

%%
%% \BibTeX command to typeset BibTeX logo in the docs
\AtBeginDocument{%
  \providecommand\BibTeX{{%
    Bib\TeX}}}


%% Rights management information.  This information is sent to you
%% when you complete the rights form.  These commands have SAMPLE
%% values in them; it is your responsibility as an author to replace
%% the commands and values with those provided to you when you
%% complete the rights form.
\copyrightyear{2025} 
\acmYear{2025} 
\setcopyright{cc}
\setcctype{by}
\acmConference[CHI '25]{CHI Conference on Human Factors in Computing Systems}{April 26-May 1, 2025}{Yokohama, Japan}
\acmBooktitle{CHI Conference on Human Factors in Computing Systems (CHI '25), April 26-May 1, 2025, Yokohama, Japan}\acmDOI{10.1145/3706598.3713296}
\acmISBN{979-8-4007-1394-1/25/04}

% These commands are for a PROCEEDINGS abstract or paper.
% \acmConference[Conference acronym 'XX]{Make sure to enter the correct
%   conference title from your rights confirmation emai}{June 03--05,
%   2018}{Woodstock, NY}
%
%%  Uncomment \acmBooktitle if the title of the proceedings is different
%%  from ``Proceedings of ...''!
%%
%%\acmBooktitle{Woodstock '18: ACM Symposium on Neural Gaze Detection,
%%  June 03--05, 2018, Woodstock, NY}
%\acmISBN{978-1-4503-XXXX-X/18/06}


%%
%% Submission ID.
%% Use this when submitting an article to a sponsored event. You'll
%% receive a unique submission ID from the organizers
%% of the event, and this ID should be used as the parameter to this command.
%%\acmSubmissionID{123-A56-BU3}

%%
%% For managing citations, it is recommended to use bibliography
%% files in BibTeX format.
%%
%% You can then either use BibTeX with the ACM-Reference-Format style,
%% or BibLaTeX with the acmnumeric or acmauthoryear sytles, that include
%% support for advanced citation of software artefact from the
%% biblatex-software package, also separately available on CTAN.
%%
%% Look at the sample-*-biblatex.tex files for templates showcasing
%% the biblatex styles.
%%

%%
%% The majority of ACM publications use numbered citations and
%% references.  The command \citestyle{authoryear} switches to the
%% "author year" style.
%%
%% If you are preparing content for an event
%% sponsored by ACM SIGGRAPH, you must use the "author year" style of
%% citations and references.
%% Uncommenting
%% the next command will enable that style.
%%\citestyle{acmauthoryear}

\graphicspath{{figs/}{figures/}{pictures/}{images/}{./}} % where to search for the images

\usepackage{acmart-taps}
\usepackage{enumitem}
\usepackage{xspace}
\usepackage{colortbl}  

\newcommand\spacer[1]{\rule[-#1]{0pt}{#1}}

\usepackage{listings}  

\def\sectionautorefname{Section}
\def\subsectionautorefname{Section}
\def\figureautorefname{Figure}

\newcommand{\bpstart}[1]{\smallskip\noindent{\textbf{#1.}}}
%\newcommand{\bpstart}[1]{\noindentparagraph{{\bf #1.}}}
\newcommand{\code}[1]{{\fontfamily{phv}\selectfont\footnotesize {#1}}}

% \newcommand{\todo}[1]{{\color{orange}\bf [todo: {#1}]}}
% \newcommand{\chenglong}[1]{{\color{purple}\bf [chenglong: {#1}]}}
% \newcommand{\steve}[1]{{\color{blue}\bf [steve: {#1}]}}

% quote coloring
\usepackage{soul}
\newcommand{\hlc}[2][yellow]{{%
                  \colorlet{foo}{#1}%
                  \sethlcolor{foo}\hl{#2}}%
}
% Define the periwinkle color  
\definecolor{periwinkle}{RGB}{240, 240, 255} 
\definecolor{urlcolor}{RGB}{24,64,127}  
\definecolor{promptcolor}{RGB}{255, 240, 240} 
\newcommand\qt[2]{\hlc[{#1}]{#2}}
\newcommand{\pquote}[1]{\qt{periwinkle}{\emph{#1}}}
\newcommand{\pprompt}[1]{\qt{promptcolor}{\emph{#1}}}

\newcommand{\new}[1]{{ {#1}}}
\newsavebox{\fmbox}
\newenvironment{smpage}[1]
{\begin{lrbox}{\fmbox}\begin{minipage}{#1}}
{\end{minipage}\end{lrbox}\usebox{\fmbox}}


% Define a custom color for the filled circle  
\definecolor{tempcolor}{RGB}{44,94,127}  
% Command for circled text  
\newcommand*\circled[1]{%  
  \raisebox{.5pt}{\textcircled{\raisebox{-.9pt} {\sf #1}}}%  
}  
  
% Command for filled circle with white text  
\newcommand*\filled[1]{%  
  \begingroup  
  \setlength{\unitlength}{1ex}%  
  \begin{picture}(2,2)  
  \put(1.1,0.75){\color{tempcolor}\circle*{2}}  
  \put(1.1,0.75){\makebox(0,0){\textcolor{white}{\sf #1}}}  
  \end{picture}%  
  \endgroup  
} 

\lstdefinelanguage{json}{  
  basicstyle=\footnotesize\sffamily,  
  breaklines=true,  
  columns=fullflexible,  
  escapeinside={||},  
  xleftmargin=0pt,  
  xrightmargin=0pt,  
  backgroundcolor=\color{white},  
  showstringspaces=false,  
  frame=none,  
  literate=  
    *{:}{{{\color{tempcolor}:}}}{1}  
     {,}{{{\color{tempcolor},}}}{1}  
     {\{}{{{\color{tempcolor}\{}}}{1}  
     {\}}{{{\color{tempcolor}\}}}}{1}  
     {[}{{{\color{tempcolor}[}}}{1}  
     {]}{{{\color{tempcolor}]}}}{1},  
  keywordstyle=\bf\color{tempcolor},  
  stringstyle=\color{red},  
  commentstyle=\color{gray},  
  morekeywords={detailed_instruction, output_fields, visualization_fields, reason}, 
}

\sloppy

%%
%% end of the preamble, start of the body of the document source.
\begin{document}

%%
%% The "title" command has an optional parameter,
%% allowing the author to define a "short title" to be used in page headers.
\title{\tool: Iterative Creation of Data Visualizations, with AI Transforming Data Along the Way}

%%
%% The "author" command and its associated commands are used to define
%% the authors and their affiliations.
%% Of note is the shared affiliation of the first two authors, and the
%% "authornote" and "authornotemark" commands
%% used to denote shared contribution to the research.
\author{Chenglong Wang}
\email{chenglong.wang@microsoft.com}
\affiliation{%
  \institution{Microsoft Research}
  \city{Redmond}
  \state{Washington}
  \country{USA}
}

\author{Bongshin Lee}
\email{b.lee@yonsei.ac.kr}
\affiliation{%
  \institution{Yonsei University}
  \city{Seoul}
  \country{Korea}
}

\author{Steven Drucker}
\email{sdrucker@microsoft.com}
\affiliation{%
  \institution{Microsoft Research}
  \city{Redmond}
  \state{Washington}
  \country{USA}
}

\author{Dan Marshall}
\email{danmar@microsoft.com}
\affiliation{%
  \institution{Microsoft Research}
  \city{Redmond}
  \state{Washington}
  \country{USA}
}

\author{Jianfeng Gao}
\email{jfgao@microsoft.com}
\affiliation{%
  \institution{Microsoft Research}
  \city{Redmond}
  \state{Washington}
  \country{USA}
}

%%
%% By default, the full list of authors will be used in the page
%% headers. Often, this list is too long, and will overlap
%% other information printed in the page headers. This command allows
%% the author to define a more concise list
%% of authors' names for this purpose.
%\renewcommand{\shortauthors}{Trovato et al.}

%%
%% The abstract is a short summary of the work to be presented in the
%% article.
\begin{abstract}
Data analysts often need to iterate between data transformations and chart designs to create rich visualizations for exploratory data analysis. Although many AI-powered systems have been introduced to reduce the effort of visualization authoring, existing systems are not well suited for iterative authoring. They typically require analysts to provide, in a single turn, a text-only prompt that fully describe a complex visualization. 
We introduce \tool (\df for short), an AI-powered visualization system designed to overcome this limitation.
\df blends graphical user interfaces and natural language inputs to enable users to convey their intent more effectively, while delegating data transformation to AI.
Furthermore, to support efficient iteration, \df lets users navigate their iteration history and reuse previous designs, eliminating the need to start from scratch each time. 
A user study with eight participants demonstrated that \df allowed participants to develop their own iteration styles to complete challenging data exploration sessions.
\end{abstract}

%%
%% The code below is generated by the tool at http://dl.acm.org/ccs.cfm.
%% Please copy and paste the code instead of the example below.
%%
\begin{CCSXML}
<ccs2012>
<concept>
<concept_id>10003120.10003145.10003151</concept_id>
<concept_desc>Human-centered computing~Visualization systems and tools</concept_desc>
<concept_significance>500</concept_significance>
</concept>
<concept>
<concept_id>10010147.10010178</concept_id>
<concept_desc>Computing methodologies~Artificial intelligence</concept_desc>
<concept_significance>500</concept_significance>
</concept>
</ccs2012>
\end{CCSXML}

\ccsdesc[500]{Human-centered computing~Visualization systems and tools}
\ccsdesc[500]{Computing methodologies~Artificial intelligence}

%%
%% Keywords. The author(s) should pick words that accurately describe
%% the work being presented. Separate the keywords with commas.
%\keywords{Visualization system, AI}
%% A "teaser" image appears between the author and affiliation
%% information and the body of the document, and typically spans the
%% page.
\begin{teaserfigure}
\centering
  \includegraphics[width=\linewidth]{figures/data-anvil-teaser.png}
    \caption{With \tool, analysts can iterate on a previous design by (1) selecting a chart from data threads and (2) providing combined natural language and graphical user interface inputs in the chart builder to specify the new design. The AI model  generates code to transform the data and update the chart. Data threads are updated with new charts for future use.}
    \label{fig:data-anvil-teaser}
\end{teaserfigure}

% \received{20 February 2007}
% \received[revised]{12 March 2009}
% \received[accepted]{5 June 2009}

%%
%% This command processes the author and affiliation and title
%% information and builds the first part of the formatted document.
\maketitle

Most modern visualization authoring tools (e.g., Charticulator~\cite{ren2019charticulator}, Data Illustrator~\cite{liu2018data}, Lyra~\cite{satyanarayan2014lyra}) and libraries (e.g., ggplot2~\cite{wickham2009ggplot2}, Vega-Lite~\cite{satyanarayan2017vegalite}) expect tidy data~\cite{wickham2014tidy-data}, where every variable to be visualized is a column and each observation is a row. When the input data is in the tidy format, authors simply need to bind data columns to visual channels (e.g., \code{Date} $\mapsto x$-axis, \code{Temperature} $\mapsto y$-axis, \code{City} $\mapsto$ color in \cref{fig:sea-atl-temp-simple}). Otherwise, they need to prepare the data, even if the original data is clean and contains all information needed~\cite{bartram2021untidy}. Authors usually rely on data transformation libraries (e.g., tidyverse~\cite{wickham2019tidyverse}, pandas~\cite{the_pandas_development_team_2023_7741580}) or separate interactive tools (e.g., Wrangler~\cite{kandel2011wrangler}) to transform data into the appropriate format. However, authors need either programming experience or tool expertise to transform data, and they have to withstand the overhead of switching between visualization and data transformation steps. The challenge of data transformation remains a barrier in visualization authoring.

To address the data transformation challenge, we explore a fundamentally different approach for visualization authoring, leveraging an AI agent. We separate the high-level visualization intent ``\emph{what to visualize}'' from the low-level data transformation steps of ``\emph{how to format data to visualize},'' and automate the latter to reduce the data transformation burden. Specifically, we support two key types of data transformations (and their combinations) needed for visualization authoring:

\begin{itemize}
    \item \textbf{Reshaping}: A variable to be visualized is spread across multiple columns or one column includes multiple variables. For example, if authors want to create a different scatter plot from the table in \cref{fig:sea-atl-temp-simple} by mapping \code{Seattle} and \code{Atlanta} temperatures to $x,y$-axes (\cref{fig:sea-atl-temp-pivot-derived}-\circled{1}), they need to first ``pivot'' the table from long to wide format, because both variables of interest are stored in the \code{Temperature} column and are not readily available.
    \item \textbf{Derivation}: A variable needs to be extracted or derived from one or more existing columns. For example, if authors want to create a bar chart to show daily temperature differences between two cities (\cref{fig:sea-atl-temp-pivot-derived}-\circled{2}) and a histogram to count the number of days which city is warmer (\cref{fig:sea-atl-temp-pivot-derived}-\circled{3}), they need to derive the temperature difference and the name of the warmer city from the two cities' temperature columns, and map them to the $y$-axis and $x$-axis, respectively, and the city name to color channels of the corresponding charts. The derivation is also needed when the variable to be visualized requires analytical computation (e.g., aggregation, moving average, percentile) across multiple rows from a column in the table. For example, to plot a line chart to visualize the 7-day moving averages of \code{Seattle} temperatures (\cref{fig:sea-atl-temp-pivot-derived}-\circled{4}), the authors need to calculate the moving average using a window function and map it to  $y$-axis with \code{Date} on $x$-axis.
\end{itemize}

\begin{figure}[t]
    \centering
    \includegraphics[width=0.95\linewidth]{figures/intro-sea-atl-temp-simple.png}
    \caption{A dataset of Seattle and Atlanta daily temperatures in 2020 (left) and a scatter plot that visualizes them by mapping Date to $x$-axis, Temperature to $y$-axis, and City to color (right).}
    \label{fig:sea-atl-temp-simple}
\end{figure}


\begin{figure*}[ht]
    \centering
    \includegraphics[width=\linewidth]{figures/intro-sea-atl-pivot-derived-plots.png}
    \caption{Visualizations created from \code{df} in \cref{fig:sea-atl-temp-simple} that require data transformation: (1) a scatter plot with Seattle and Atlanta temperatures on $x,y$-axes, (2) a bar chart to visualize the temperature difference between the two cities, (3) a histogram to count the number of days each city being warmer, and (4) a smoothed line chart that shows the 7-day moving averages of Seattle temperature.}
    \label{fig:sea-atl-temp-pivot-derived}
\end{figure*}


In this paper, we introduce Data Formulator, an interactive visualization authoring tool that embodies a new paradigm, \emph{concept binding}. To create a visualization with Data Formulator, authors provide their visualization intent by binding data concepts to visual channels. Upon loading of a data table, existing data columns are provided as known data concepts. When the required data concepts are not available to author a given chart, the authors can create the concepts: either using natural language prompts (for derivation) or by providing examples (for reshaping). Data Formulator handles these two cases differently, with different styles of input and feedback, and we provide a detailed description of how they are handled in \cref{sec:user-experience}. Once the necessary data concepts are available, the authors can select a chart type (e.g., scatter plot, histogram) and map data concepts to desired visual channels. If needed, Data Formulator dispatches the backend AI agent to infer necessary data transformations to instantiate these new concepts based on the input data and creates candidate visualizations. Because the authors' high-level specifications can be ambiguous and Data Formulator may generate multiple candidates, Data Formulator provides feedback to explain and compare the results. With this feedback, the authors can inspect, disambiguate, and refine the suggested visualizations. After that, they can reuse or create additional data concepts to continue their visualization authoring process.

We also report a chart reproduction study conducted with 10 participants to gather feedback on the new concept binding approach that employs an AI agent, and to evaluate the usability of Data Formulator. After an hour-long tutorial and practice session, most participants could create desired charts by creating data concepts—both with derivation and reshaping transformations. We conclude with a discussion on the lessons learned from the design and evaluation of Data Formulator, as well as important future research directions.




% \begin{figure}
%     \centering
%     \includegraphics[width=0.5\linewidth]{figures/global-energy-data.png}
%     \caption{ We refer to this table as \code{df} in this section.}
%     \label{fig:global-energy-datatable}
% \end{figure}


% \begin{figure}[t]
%     \centering
%     \includegraphics[width=0.5\linewidth]{figures/global-energy-basics.png}
%     \caption{Basic charts from \autoref{fig:global-energy-datatable} show the distributions of Electricity from renewables and CO$_2$ emissions from each country over time}
%     \label{fig:global-energy-basics}
% \end{figure}
% \begin{figure}[t]
%     \centering
%     \includegraphics[width=0.5\linewidth]{figures/global-energy-facets.png}
%     \caption{A faceted line charts comparing Electricity from three sources. We need to unpivot \code{df} in \autoref{fig:global-energy-datatable} to surface \code{Energy Source} and \code{Electricity}.}
%     \label{fig:global-energy-facets}
% \end{figure}

% \begin{figure*}[t]
%     \centering
%     \includegraphics[width=0.95\linewidth]{figures/global-energy-renewable-percentage.png}
%     \caption{The creation and refinements of the visualizations about renewable energy percentage trends. Despite these charts are closely related, challenging data transformations involving aggregation, filtering, table merge and calculation are required in these refinement steps (tables 1-3).}
%     \label{fig:global-energy-renewable-percentage}
% \end{figure*}


% \begin{figure*}[t]
%     \centering
%     \includegraphics[width=\linewidth]{figures/data-anvil-overview.png}
%     \caption{Data Anvil overview. The user creates visualizations by providing fields (drag-and-drop from \textbf{Data Fields} or type in new fields) and NL instructions to \textbf{Chart Builder} and asks AI agents to formulate. \textbf{Data View} and \textbf{Data Threads}  show the derived data and the derivation history. Next, the user can revise, refine or create new charts from existing ones by providing follow-up instructions in \textbf{Local Threads} and \textbf{Chart Builder}.}
%     \label{fig:data-anvil-overview}
% \end{figure*}

\begin{figure*}[t]
\includegraphics[width=\linewidth]{figures/example-renewable-energy.png}
    \caption{An analyst explores electricity from different energy sources, renewable percentage trends, and country rankings by renewable percentages using a dataset on CO$_2$ and electricity for 20 countries (2000-2020, table 1). The analyst creates five data versions in three branches to support different chart designs. \df allows users to manage iteration directions and create rich visualizations using a blended UI and natural language inputs.}
    \label{fig:example-analysis-session}
\end{figure*}

\section{Illustrative Scenarios}
\label{sec:illustartive-scenarios}

In this section, we describe scenarios to illustrate users’ experiences for creating a series of visualizations to explore global sustainability from a dataset of 20 countries' energy generation from 2000 to 2020. The initial dataset, shown in \autoref{fig:example-analysis-session}-\circled{1}, includes each country's energy produced from three sources (fossil fuel, renewables, and nuclear) each year and annual CO$_2$ emission value (the CO$_2$ emission data only ranges from 2000 to 2019). 
We compare different experiences and skills required for a data analyst, Megan, to complete the analysis session shown in \autoref{fig:example-analysis-session} with different tools, computational notebooks versus \df.

\begin{figure*}[t]
    \centering
    \includegraphics[width=\linewidth]{figures/data-anvil-UI.png}
    \caption{\df overview. Users create visualizations by providing fields (drag-and-drop or type) and NL instructions to the Chart Builder, delegating data transformation to AI. \textbf{Data View} shows derived data. Users navigate data history and select contexts for the next iteration using (the thread in use is displayed as \textbf{local data threads}). They refine or create new charts by providing instructions in \textbf{Chart Builder}. The main panel provides pop-up windows to inspect code, explanations, and chat history.}
    \label{fig:data-anvil-overview}
\end{figure*}


\bpstart{Exploration with computational notebooks}
To complete the analysis in a computation notebook, Megan can use R libraries \textsf{ggplot2} and \textsf{tidyverse}. To use \textsf{ggplot2} to create charts, Megan needs to make sure that all data fields to be visualized on visual channels (e.g., $x,y$-axes, color, facet) are columns in the input data, thus, Megan uses \textsf{tidyverse} to transform data when needed.

\autoref{fig:example-analysis-session} shows Megan's data analysis session with three branches. She starts with two basic line charts (chart \circled{1}-A,B) showing renewable energy and CO2 emission trends. Megan observes that many countries' CO2 emissions have increased despite increased renewable energy use, prompting her to create a faceted line chart (chart \circled{2}) and visualize renewable energy percentage trends (chart \circled{3}). Discovering that renewable percentage is a better indicator for global sustainability trends, Megan explores two directions: creating a line chart of countries' renewable percentage ranks (chart \circled{4}) and highlighting the top 5 CO2 emitters' trends (chart \circled{5}) compared to global median values (chart \circled{6}). Throughout the process, Megan backtracks several times to fork new branches from a previous version of data (e.g., charts \circled{2} to \circled{3}, and \circled{4} to \circled{5}) and reuses existing results to create new charts (e.g., chart \circled{6} from \circled{5}).

Implementing these charts requires considerable data preparation efforts. While basic charts can be created by mapping existing data fields to visual channels (e.g., \code{Year}$\rightarrow x$, \code{Electricity from renewables (Twh)}$\rightarrow y$, \code{Entity}$\rightarrow$\code{color} for chart \circled{1}-A), more complex charts (\circled{3}-\circled{6}) require different data transformations. For example, Megan needs to reshape the table with \code{pivot\_longer} to merge energy sources into a new field \code{Electricity} for the $y$-axis (chart \circled{2}); to rank countries by renewable percentage (chart \circled{4}), she partitions the data by year and uses \code{rank}; for charts \circled{5} and \circled{6}, she computes the global median using aggregation and merges the results with the previous table to surface all necessary fields.


\begin{comment}
\subsection{Exploration with computational notebooks}

Heather is an analyst who is proficient with a computational notebook and R libraries, \textsf{ggplot2} and \textsf{tidyverse}. Because \textsf{ggplot2} expects all data fields to be visualized on visual channels (e.g., $x,y$-axes, color, facet) to be columns in the input data, Heather uses \textsf{tidyverse} for data transformation.

\bpstart{Basic charts} To start, Heather wants to visualize the amount of electricity produced from renewables per country over the years with a line chart to see ``{if our planet is operating in a sustainable fashion.}'' Since the input data (table-\circled{1}) includes all required fields, Heather creates the line chart with ease, by mapping columns \code{Year}$\rightarrow x$, \code{Electricity from renewables (Twh)}$\rightarrow y$ and \code{Entity} $\rightarrow$\code{color} (chart \circled{1}-A). She then creates another line chart for CO$_2$ emission trends, mapping \code{CO$_2$ emissions (kt)} to the $y$ axis (chart \circled{1}-B). Heather is puzzled that China, the country with considerably increased use of renewable energy, also has the biggest increase in CO$_2$ emissions. This is counterintuitive because renewables themselves would not cause CO$_2$ emission increases. Thus, Heather decides to delve deeper. 

\bpstart{Renewable energy versus other sources} Heather suspects the CO$_2$ emission increase is caused by a surge of fossil fuel consumption. To compare fossil fuel usage against renewables, she wants a faceted line chart that shows electricity from each energy source side by side (chart \circled{2}). To create the chart, Heather needs to have a data table with columns--\code{Year}, \code{Electricity}, \code{Entity}, and \code{Energy Source}--and map the columns to $x,y$, \textsf{color}, and \textsf{facet}, respectively so that the chart is divided into subplots based on values from the \code{Energy Source} column. Because table \circled{1} stores electricity values across three columns in the wide format, Heather unpivots table \circled{1} into the long format, to fold specified column names into values in the \code{Energy Source} field and corresponding values into the \code{Electricity} field. She then creates the desired chart \circled{2} with the transformed data \circled{2}, and verifies her assumption: despite the increase in renewables usage, the usage of fossil fuel also grows significantly, leading to CO$_2$ emission increase. This motivates Heather to explore renewable trends by visualizing trends of the \emph{percentage} of electricity from renewables over all three resources. %as opposed to the basic chart.

\bpstart{Renewable energy percentage and ranks} To visualize renewable energy percentage, Heather goes back to table \circled{1} to derive a new column \code{Renewable Percentage}, by dividing \code{Electricity from renewables (TWh)} from the total produced electricity for each country per year. With the new data \circled{3}, Heather visualizes the renewable percentage trends in chart \circled{3}, which shows that the percentage increase is slower than their absolute value increase (as shown in chart \circled{1}). 

Because many countries share similar renewable percentage, it is quite difficult to compare different countries' trends. Heather thus decides to create a visualization of countries' renewable percentage ranks to complement the existing charts. To calculate ranks for each country among others per year, Heather uses a window function on table \circled{3} to partition the table based on \code{Year}, and apply the \textsf{rank()} function to \code{Renewable Percentage} to derive a new column \code{Rank}. With \code{Rank} mapped to $y$-axis, chart \circled{4} allows Heather to clearly examine how different countries' ranks change in the last two decades; for example, Germany and UK are the two top ranked countries emerge from the bottom pack in 2000.

\bpstart{Renewable trends from top CO2 emitters} Finally, Heather wants to focus on renewable percentage trends from top CO$_2$ emission countries, which make most influences to global sustainability. Despite table~\circled{3} contains all columns to be visualized, Heather needs to filter it based on the countries' CO2 emission. To do so, Heather goes back to table \circled{1} to aggregate each country's total CO$_2$ emission, sort it and find top five. Heather then uses this intermediate result to filter table~\circled{3} to obtain renewable percentage from top five CO2 emitters (shown as table~\circled{5}) and creates chart~\circled{5}.

From this chart, it is clear that top CO$_2$ emitters are indeed heading in the right direction towards sustainability, despite total CO$_2$ emissions still increasing with total energy produced also increasing each year. To publish this visualization, Heather decides to add an annotation to the plot with the median global renewable percentage. On top of table~\circled{5}, Heather appends the median renewable percentage each year calculated from table~\circled{3} and includes a new column \code{Global Median?}, used as a flag to assist plotting so that global median can be colored in a different opacity. Chart~\circled{6} shows the final result, by including \code{Global Median} as an \code{Entity} and mapping \code{Global Median?}$\rightarrow$\textsf{opacity}, median renewable percentage is visualized along other countries in a different opacity. Heather is satisfied with the results and concludes the session. 
\end{comment}

%Since Heather is programming in Jupyter notebook, she could manage iterations between data and charts in the same context: she could sidetrack to create a faceted line chart (\autoref{fig:global-energy-facets}) and edit code incrementally to refine the design (\autoref{fig:global-energy-renewable-percentage}) while retaining all versions of data and charts.  Second, data transformation would be a significant barrier if Heather is not an experienced data scientist. Even though many of iteration steps are semantically close, they follow challenging and diverse data transformations, including filtering, reshaping, aggregation and table merge. Heather further needs proficiency with implementation details like \code{reset\_index(), ignore\_index} to piece together these transformations.



% \begin{figure}[t]
%     \centering
%     \includegraphics[width=\linewidth]{figures/data-anvil-explanation.png}
%     \caption{Data Anvil's AI agent generates a NL explanation from the code to explain the data transformation. Users can also view code directly.}
%     \label{fig:data-anvil-explanation}
% \end{figure}


% \begin{figure}[t]
%     \centering
%     \includegraphics[width=0.95\linewidth]{figures/data-anvil-correct-errors.png}
%     \caption{Unclear instructions can lead to undesired results, the user can (1) provide a followup instruction to instruct AI to correct the error, or (2) backtrack to the previous step, revise  and re-run data formulation.}
%     \label{fig:data-anvil-handle-errors}
% \end{figure}



\bpstart{Exploration with \df}
Using \df to complete the same analysis session, Megan's experience is quite different. Instead of transforming data and creating visualizations with code, Megan's main task is to describe visualization goals with UI interactions and NL inputs and ask the AI model to realize them.

Megan starts with basic line charts to visualize trends of electricity from renewables (\autoref{fig:example-analysis-session}-\circled{1}A). Since all three required fields are available from the input data, Megan simply selects the chart type ``line chart'' in the encoding shelf and drags and drops fields to their corresponding visual channels (\autoref{fig:data-anvil-basics-facets}-\circled{1}). \df then generates the desired visualization. To visualize the CO$_2$ emission trends, Megan swaps the $y$-axis encoding with \code{CO2 emissions (kt)}$\rightarrow y$.

\begin{figure*}[t]
    \centering
    \includegraphics[width=\linewidth]{figures/data-anvil-basics-facets.png}
    \caption{Experiences with \df: (1) creating the basic renewable energy chart using drag-and-drop to encode fields;  (2 and 3) creating charts requiring new fields by providing field names and optional natural language instructions to derive new data.}
    \label{fig:data-anvil-basics-facets}
\end{figure*}

Megan now needs to create the faceted line chart to compare electricity from all energy sources, which requires new fields \code{Electricity} and \code{Energy Source}. With \df, Megan can specify the chart using new data fields and NL instructions in the chart builder (\autoref{fig:data-anvil-overview}-2) and ask the AI to transform the data. 
As \autoref{fig:data-anvil-basics-facets}-\circled{2} shows, Megan first drags and drops existing fields \code{Year} and \code{Entity} to the $x$-axis and \textsf{color}, respectively. Then, she types in the names of new fields \code{Electricity} and \code{Energy Source} in the $y$-axis and \textsf{column}, respectively, to indicate to the AI agent that she expects two new fields to be derived for these properties. Finally, Megan provides an instruction, ``compare electricity from all three sources,'' to further clarify the intent and clicks the formulate button.
To create the chart, \df first generates a Vega-Lite spec skeleton from the encoding (to be completed based on information from the transformed data). It then summarizes the data, encodings, and NL instructions into a prompt to ask an AI model to generate data transformation code to prepare the data that fulfills all necessary fields, which is then used to instantiate the chart skeleton. After reviewing the generated chart and data, Megan is satisfied and moves to the next task. 
\df also updates data threads (\autoref{fig:data-anvil-overview}-\circled{5}) with the newly derived data and chart. With data threads, Megan can switch the iteration contexts to instruct the AI model to create a new chart either from scratch or reusing a previous result. 

\begin{figure*}[t]
    \centering
    \includegraphics[width=\linewidth]{figures/data-anvil-refinement.png}
    \caption{Iteration with \df: (1) provide an instruction to filter the renewable energy percentage chart by top CO$_2$ countries, (2) update the chart with \code{Global Median?} and instruct \df to add the global median alongside the top 5 CO$_2$ countries' trends, and (3) move \code{Global Median?} from \textsf{column} to \textsf{opacity} to update the chart design without deriving new data.}
    \label{fig:data-anvil-refinement}
\end{figure*}

Megan proceeds to visualize renewable energy percentage. Although it requires a different data transformation, Megan's experience is similar to the previous one: she drags-and-drops \code{Year} and \code{Entity} to $x$-axis and \textsf{color} (\autoref{fig:data-anvil-basics-facets}-\circled{3}), and enters the name of the new field ``\code{Renewable Energy Percentage}'' on the $y$-axis. Since Megan believes the field names are self-explanatory, she formulates the new data without an additional NL instruction. \df generates the desired visualization (\autoref{fig:data-anvil-refinement}-\circled{1}). To visualize the countries' renewable percentage ranks, building on the previous data, Megan adds a new field ``Rank'' to the $y$-axis and provides a short instruction. Because Megan builds the new chart on top of the previous data (note that in \autoref{fig:data-anvil-basics-facets}-\circled{3}, the chart builder box is positioned under the previous \textsf{table-42} as opposed to \textsf{energy.csv}), the AI model has more contextual information to correctly derive the renewable percentage rank (\autoref{fig:example-analysis-session}-\circled{4}) despite Megan's simple inputs.

Next, to visualize the renewable percentage trends of the top five CO$_2$ emitting countries, Megan decides to build on a previous chart to avoid creating a verbose prompt from scratch.
Megan first uses data threads (\autoref{fig:data-anvil-overview}-\circled{5}) to locate renewable percentage chart and opens it in the main panel. On top of that, Megan provides a new instruction below the local data thread, {``show only top 5 CO2 emission countries' trends,''} and clicks the ``derive'' button (\autoref{fig:data-anvil-refinement}-\circled{1}). \df updates the previous code to include a filter clause to produce the new data and visualization (\autoref{fig:data-anvil-refinement}-\circled{2}). Finally, to annotate the chart with global median trends, Megan forks a branch by copying the previous chart, as the new chart requires different encodings (and she wants to keep both visualizations available). Megan updates the visual encoding by (1) typing in a new field name \code{Global Median?} for \textsf{column} and (2) providing the edit instruction ``include global median as an entity'' (\autoref{fig:data-anvil-refinement}-\circled{2}). Once she clicks the derive button, \df generates the new chart (\autoref{fig:data-anvil-refinement}-\circled{3}). Upon inspection, Megan prefers to change the visualization type, with global average rendered in a different opacity as opposed to a different subplot. Since these two charts require the same data fields, Megan doesn't need to interact with the AI model --- she can directly update the design through the UI:  first selecting a new chart type ``custom line'' (which exposes more chart properties than the basic line chart) and moving \code{Global Median?} to the \textsf{opacity} channel. With all desired charts created, Megan concludes the analysis session. \autoref{fig:data-anvil-overview}-\circled{3} shows all the data threads from Megan.

\bpstart{Comparison of experiences}  These two tools offer different experiences and skill requirements for Megan to execute the analysis. However, both enable her to iteratively refine exploration goals and explore different branches to uncover insights.

The main difference between the two experiences is data transformation. In computation notebooks, Megan needs to prepare data for design updates, even seemingly small ones (e.g., charts-\circled{3} and \circled{5}). She must understand the data shape required and apply the correct transformations (e.g., unpivot for table~\circled{2}, join and union for table~\circled{6}). Proficiency in data transformation is essential for creating rich visualizations. In \df, Megan specifies high-level chart designs, and the AI implements the transformations. Regardless of the underlying data transformations, she conveys her intents uniformly through visual encodings (UI) and natural language inputs. Because Megan can use the shelf-configuration UI to specify chart design, the supplementary NL instruction is straightforward. Though Megan doesn't write code, \df provides artifacts like generated data, charts, and code with natural language explanations for her to review. By lowering the implementation skill barrier, \df allows users to focus more on analysis planning and reasoning.

Computation notebooks naturally support reuse. Megan can copy-edit previous code snippets or reuse variables to build new charts.  In \df, Megan directs the analysis using data threads. Megan can easily review the history and select previous results to instruct the AI model to create new charts from those contexts. This simplifies instructions to incremental updates, and the AI reuses previous outputs to avoid mistakes. If undesired results occur, she can backtrack and revise inputs using data threads (\autoref{fig:data-anvil-overview}-\circled{3}). Iteration isn't as easy with a chat-based tool. Iteration isn't as easy with a chat-based tool, where verbose prompts are needed to guide the AI and avoid unrelated histories.
\section{System Design}
\label{sec:system_design}
In this section, we present \df's system design.
First, to enable users to specify their intent using multiple paradigms (shelf-configuration UI and NL inputs) \df \textbf{decouples chart specification from data transformation}, solving them with template instantiation and AI code generation respectively.
Second, to support reuse, \df organizes \textbf{the iteration history as data threads with data as first-class objects}.
 This enables users to either locate a chart from a different branch and follow up or quickly revise and rerun the most recent instructions leading to the current chart. 
We will next detail how we implement these designs and explain how additional features help users understand AI-generated results.

\subsection{Composing charts from multi-modal inputs}
\autoref{fig:multi-modal-ui-approach} shows how \df decouples chart design and data transformation to support blended input methods. Given a user specification, \df generates the desired chart in three steps: (1) generate a Vega-Lite specification from the selected chart type, (2) compile a prompt and delegate data transformation to the AI, and (3) instantiate the Vega-Lite specification with the generated data.

\begin{figure*}[t]
    \centering
    \includegraphics[width=1\linewidth]{figures/architecture-multi-modal-UI.png}
    \caption{\df's workflow: (1) \df generates a Vega-Lite spec skeleton based on user specifications and chart type. (2) If new fields (e.g., \code{Rank}) are required, \df prompts its AI model to generate data transformation code. (3) The Vega-Lite skeleton is then instantiated with the new data to produce the desired chart.}
    \label{fig:multi-modal-ui-approach}
\end{figure*}

\bpstart{Chart specification generation} \df adopts a chart type-based approach to represent visualizations, supporting five categories of charts: scatter (scatter plot, ranged dot plot), line (line chart, dotted line chart), bar (bar chart, stacked bar chart, grouped bar chart), statistics (histogram, heatmap, linear regression, boxplot) and custom (custom scatter, line, bar area, rectangle where all available visual channels are exposed). Each chart type is represented as a Vega-Lite template with a set of predefined visual channels, including position ($x$, $y$), legends (\textsf{color}, \textsf{size}, \textsf{shape}, \textsf{opacity}), and facet (\textsf{column}, \textsf{row}) that are shown to the user in the chart builder. For example, a line chart is represented as a Vega-Lite template \textsf{\{ "mark": "line", "encoding" : \{ "x": null, "y": null, "color": null, "column": null, "row": null\}\}}, and when the user selects line chart, channels $x$, $y$, \textsf{color}, \textsf{column}, and \textsf{row} are displayed in the chart builder. Chart type-based design enable \df to support predefined layered charts (e.g., ranged dot plot composed from line and scatter, \autoref{fig:template-instantiation}). Additional chart types (e.g., bullet chart) can be supported by adding Vega-Lite templates with respective channels to the library.

\begin{figure*}
    \centering
    \includegraphics[width=1\linewidth]{figures/template-instantiation.png}
    \caption{\df converts user encodings into a Vega-Lite specification, which is combined with AI-transformed data to visualize country ranks in 2000 and 2020.}
    \label{fig:template-instantiation}
\end{figure*}
    

As the user inputs fields into the chart builder, either by dragging and dropping it from existing data fields or by typing in new fields they wish to visualize, \df instantiates the Vega-Lite template with provided fields. For example, as shown in \autoref{fig:multi-modal-ui-approach}-\filled{1}, when the user drags \code{Year}$\rightarrow x$, \code{Entity}$\rightarrow y$ and types \code{Rank} in $y$, the line chart template mentioned above is instantiated with provided fields: if the field is available in the current data table, both field name and encoding type are instantiated (e.g., \code{Year} with the temporal type), otherwise the encoding type is left as a ``<placeholder>'' to be instantiated later when data transformation completes. 
The shelf-configuration saves users efforts from writing prompts to explain complex chart designs. For example, to create a ranged dot plot--layered chart composed of scatter and line charts--the user only needs to fill the required fields in the UI. \df then populates corresponding fields in the predefined chart template (\autoref{fig:template-instantiation}). 

\bpstart{Data transformation with AI} From the chart builder, \df assembles a prompt and queries an LLM to generate python code to transform data. The data transformation prompt contains three segments: the system prompt, the data transformation context and the goal (illustrated \autoref{fig:multi-modal-ui-approach}-\filled{2}).

The \textbf{system prompt} describes the role of the LLM and the output format. Besides generic role descriptions (i.e., LLM as a data scientist for data transformation), the system prompt guides the LLM to solve the data transformation task in two steps. First, the LLM should refine the user's goal and output as a JSON object that elaborates intermediate and final fields to be computed from the original data. Then, the LLM should generate a python snippet following a provided template. The system prompt ends with an input-output example that illustrates the process. The design rationale behind the ``goal refinement'' step is to allow the LLM to reason about any potential discrepancy between users' provided fields and their instruction (e.g., users may ask about color by energy type but didn't put ``energy type'' on the color encoding) and determine the final list of fields to be computed.
\df then assembles \textbf{context prompts} that illustrate the data to be transformed, explaining the data fields by showing the data type and example values for each field, along with sample table rows. The data context provides valuable information related to data formats (such as data types, string formats, and whether columns contain null values) to the LLM, ensuring that the generated transformation code is executable on the given data. When a chart is specified based on previous results, the dialog history between \df and the LLM, including user instructions and previously generated code, is appended in context. This way, even if users' follow-up prompts is short, the grounded contexts help the model understand user intent and reuse previously generated code.
Finally, \df assembles a \textbf{goal prompt}, combining the NL instruction provided in the text box and field names used in the encodings. When users skip an NL instruction (\autoref{fig:data-anvil-basics-facets}-\filled{3}), the instruction part is left blank. This goal will be refined by the LLM (i.e., based on the system prompt) before attempting to generate the data transformation code.
With the full input, \df prompts the LLM to generate a response. Below shows the LLM's refined goal for the task in \autoref{fig:multi-modal-ui-approach}, and the generated code is shown in \autoref{fig:multi-modal-ui-approach}-\filled{2}.

\begin{imageonly}
\begin{center}
\begin{smpage}{0.95\linewidth}
\begin{lstlisting}[language=json]  
{ "detailed_instruction": "Calculate the percentage of electricity generated from renewables for each country per year. Then, rank the countries by their renewable percentage for each year.",
 "output_fields": ["Year", "Entity", "Renewable_Percentage", "Rank"],
 "visualization_fields": ["Year", "Rank", "Entity"],
 "reason": "To rank countries by their renewable percentage, we need to calculate the renewable percentage for each country per year and then determine the rank based on this percentage." }
\end{lstlisting}
\end{smpage}
\end{center}
\end{imageonly}

\smallskip

\df then runs the code on the input data. If the code executes without errors, the output data is used to instantiate the Vega-Lite script generated in the previous step. This is done by first inferring semantic types of newly generated columns (to determine their encoding type), and then assembling the data with the script to render the visualization (\autoref{fig:multi-modal-ui-approach}-\filled{3}). The generated code sometimes causes runtime errors due to an attempt to use libraries that are not imported, references to invalid columns names, or incorrect handling of \code{undefined} or \code{NaN} values. When such errors occur, \df tries to correct the errors by querying the LLM with the error message and a follow-up instruction to repair its mistakes~\cite{olausson2023self,chen2023teaching}. The visualization is generated when repair completes. \df updates the data threads upon creating the chart.

% , the user first specifies chart encoding with a shelf-configuration UI; then, they provide an NL instruction to elaborate the intent. These inputs are grounded in the contexts (input data, and authoring history) into a prompt that delegates the implementation task to the data agent. As demonstrated in \autoref{fig:data-anvil-basics-facets}, in the encoding shelf, the user can refer to \emph{future} fields by providing their names besides existing data fields, and this interaction provide both the schema of the expected output data and a precise specification of the chart. Using NL inputs, the user has the freedom to provide either a declarative description that illustrate the relation/semantics of the expected visualization (e.g., ``compare electricity from all sources'' as demonstrated in \autoref{fig:data-anvil-basics-facets}-\circled{2}), or an imperative instruction that explains how the computation should be done (e.g., ``pivot the table'' for the same task). 

%With the user specification, Data Anvil first generates a Vega-Lite specification with placeholder data that reflects users' chart design; it then compiles a grounded prompt consisting of (1) system prompt that describes Data Agent's objective (i.e., to fill a Python template with pandas library based), (2) few-shot examples that demonstrate the code generation tasks~\cite{brown2020language}, (3) a summary of the input data that the code should operate on, (4) information of previous visualization iterations (if the new task is iterated from an existing one), and (5) the expected output data schema (extracted from UI input) and NL instructions. 
%For example, given the user specification in \autoref{fig:data-anvil-basics-facets}-\circled{2}, the following Vega-Lite spec and prompt are generated:
% \begin{center}
% \begin{smpage}{0.95\linewidth}
% \begin{minted}[fontfamily=helvetica,fontsize=\small]{json}
% { "mark": "line", "encoding" : { "x": {"field": "Year", "type": "quantitative"}, "y": {"field": "Electricity"},  "color": {"field": "Entity", "type": "nominal"}, "column": {"field": "Energy Source"}, } }
% \end{minted}
% \begin{minted}[fontfamily=helvetica,fontsize=\small]{python}
% # You are a data scientist to help user to transform data. Create a python function based off the [CONTEXT], [TEMPLATE] and [GOAL]...
% #... (more prompt, summary of data and dialog, and examples are omitted)
% import pandas as pd
% def transform_data(df):
%     # complete the template here
%     return transformed_df
% # [GOAL] Expected fields ["Year", "Electricity", "Entity", "Energy Source"]
% # [INSTRUCTION] compare electricity from all sources
% \end{minted}
% \end{smpage}
% \end{center}

%The prompt is then provided to the data agent to generate the data transformation code, which produces an output data to instantiate the Vega-Lite spec to render the visualization. Benefiting from the expressiveness of Vega-Lite, Data Anvil supports bespoke charts that can be composed from Vega-Lite marks (point, line, bar, area, etc.), encodings ($x$, $y$, color, shape, size, column, row) and layering. With a generic data function template, the data agent supports a wide range of transformation idioms including table join, union, reshaping, aggregation, moving average, and ranking. Data Anvil can support such expressive language without compromising reliability thanks to LLMs' code generation capability and the user's rich inputs collected via the multi-modal UI.

%The multi-modal interface has the following benefits. First, the shelf-configuration UI allows precise specification of the chart intent with user-familiar experience and naturally supports visualizations types that can be represented as Vega-Lite templates (Data Anvil supports charts composed from point, line, bar and area marks as supported by Vega-Lite). Without it, users need to specify in much more verbose NL instructions. Then, NL instructions provide an opportunity for users to clarify their intent that cannot be conveyed in chart encoding, without which the data agent would compromise its reliability since it has to guess unspecified user intent. Because the user can specify the task more precisely, data agent can employ a more generic template (as opposed to a more constrained template that limits data transformation to a smaller set of operators like prior work~\todo{cite}). This enable Data Anvil to handle a wide range of data transformation idioms including table join, union, reshaping, aggregation, moving average, ranking and more, as demonstrated in \autoref{sec:illustartive-scenarios}. The supports of diverse data transformation operators and bespoke charts are essential to enable Data Anvil create rich visualizations.

% Without the UI input, the user would have to provide a much more verbose instruction in NL about the visual encoding and data schema to achieve the same level of control; and without the opportunity for users to explain their task with NL input, AI would have to guess meanings of new fields, which would compromise system's reliability. For example, given the table \autoref{fig:global-energy-datatable}, the user can create a bar chart with \code{Entity} $\mapsto x$ and \code{Difference} $\mapsto y$ with the instruction ``show me the difference of CO$_2$ emission between 2019 and 2000 for each country'', here, the UI input provides chart specification, the output table schema \code{[Entity, Difference]}, and NL instruction is essential to elaborate the semantics.

\subsection{Data threads}

\begin{figure*}
    \centering
    \includegraphics[width=1\linewidth]{figures/data-threads-design.png}
    \caption{Data threads and local data threads (right). Users can select previous data or charts to create new branches, and the AI reuses code for new transformations based on user instructions. The local data thread offers shortcuts to (1) rerun the previous instruction, (2) issue a follow-up instruction, or (3) expand the previous card to revise and rerun the instruction.}
    \label{fig:data-threads-design}
\end{figure*}

Data threads visualize the analyst's interaction history with AI, allowing the analyst to control the iteration direction by selecting which data or chart the AI model should use to generate new charts.
In data threads, each node represents a version of the data, and these nodes are connected by edges that represent the user's instructions provided to the AI model for data transformation. Visualizations are attached to the data from which they were created. Centering the iteration history around data benefits user navigation because it reflects the sequence of user actions in creating these new data. 

When a user issues a follow-up instruction from an existing data or chart, \df provides the previous conversation history to the AI and instructs it to rewrite the code towards new goals. Each time the user forks a new branch using data threads, the authoring context switches automatically and is highlighted in the main panel for the user's awareness. This way, the AI model minimizes the risk of incorrectly using information from other branches for data transformation. As shown in \autoref{fig:data-threads-design}, the code and the conversation history are attached to each data node. In our design, when the user issues a follow-up instruction, the AI model generates new code by updating the previous code (which may involve additions, deletions, or both) to achieve the user's goal. This ensures that the code always takes the original data as the input, with all information accessible. This way, whether the user wants to update the data (e.g., {``now, calculate the average rank for each country''}), revise the previous computation (e.g., {``also consider nuclear as renewable energy''}) or create alternatives (e.g., {``rank by CO2 instead''}), the AI model can achieve these tasks as it has access to the full dialog history and the complete dataset. Note that an alternative design where we only pass current data to the AI model and ask it to write a new code to further transform it (i.e., reusing the data as opposed to reusing the computation leading to the data) would not be ideal. With  access to only the current data, this approach cannot handle ``backtracking'' or ``generating an alternative design'' styles of instructions effectively.

During iteration, analysts need to both (1) switch to different data or a chart far from the current one to explore a different direction and (2) perform quick follow-ups or revisions of the latest instruction based on the latest data. To accommodate these different needs, \df presents both global data threads and local data threads.
For global navigation, the key challenge is to help the user distinguish the desired content from others. To address this, data threads are located in a separate panel with previews of data, instructions, and charts to assist navigation (\autoref{fig:data-anvil-overview}). This supports users' differing navigation styles, whether they want to navigate by provenance (i.e., using instruction cards to locate desired data) or by artifacts (i.e., using visualization snapshots to recall data semantics). Once the user locates the desired data, they can click and open a previous chart, displaying it in the main panel. Additionally, they can create a new chart directly from the data~\autoref{fig:data-threads-design}-\filled{1}. In contrast, the local data thread is designed as part of the main authoring panel (\autoref{fig:data-anvil-overview}). It features a much-simplified view (i.e., hiding other visualizations created in this thread) to display a copy of the current thread in use. The main goals of the local data thread are to provide users with awareness of the current iteration context (so they don't need to cross-reference between the chart builder and the data threads panels) and to offer shortcuts for quick revisions of recently created charts.
As shown in \autoref{fig:data-threads-design}, the user can perform three types of revision tasks with local data threads: rerun the previous instruction (e.g., when the AI produces an incorrect result and they would like to quickly retry, \filled{2}), provide a follow-up instruction to refine the data (\filled{3}), and quickly open the previous instruction to modify and rerun the command (\filled{4}).

\begin{figure*}
    \centering
    \includegraphics[width=0.75\linewidth]{figures/code-explanation.png}
    \caption{\df provides explanations of the code generated by AI to assist users understand the data transformation. This example is the explanation of the code behind table-56 in \autoref{fig:data-threads-design}.}
    \label{fig:code-explanation}
\end{figure*}

\subsection{Assisting user to inspect and style charts} 

As an AI-powered tool, \df allows users to verify AI-generated results and resolve AI's mistakes. It displays the transformed data and the visualization in the main panel and enables users to inspect generated code, its explanation, and the raw chat history through pop-up windows (\autoref{fig:data-anvil-overview}). This design accommodates various user verification styles~\cite{wang2021falx,gu2023analysts} such as viewing high-level correctness from the chart, inspecting corner cases in the data, examining the transformation output, and understanding the transformation process through the code. \df utilizes a code explanation module to help users understand the code, querying the AI model to translate code into step-by-step explanations. \autoref{fig:code-explanation} shows the explanation for the code behind table-56 in \autoref{fig:data-threads-design}. Expert users who would like to directly view the raw chat history between \df and the AI model (e.g., to inspect the LLM's raw reasoning process) can access this information from the ``view chat history'' pop-up window.
Note that despite that data transformations generated in the later iteration stages can be complex, users can verify its correctness against its predecessor because \df users create visualizations incrementally. This lowers users' verification efforts, as found in our study in \autoref{sec:evaluation}. To fix errors, users can take advantage of the data thread's iterative mechanism to rerun, follow up, or revise instructions. 

Benefiting from the decoupled chart specification and data transformation processes, when users want to update visualization styles (e.g., change color scheme, change sort order of an axis, or swap encodings) that do not require additional data transformation, they can directly perform edits in the chart builder. By updating channel properties or swapping encoded fields, these updates are directly reflected in the Vega-Lite script and rendered in the main panel. Unlike interactions with AI, which may have a slightly delayed response time, this approach allows users to achieve quick and precise edits with immediate visual feedback to refine the design.



\subsection{Implementation} \df is a React application with a python server for data transformation. \df has been tested with OpenAI models including GPT-3.5-turbo, GPT-4, GPT-4o, and GPT-4o-mini. We used GPT-3.5-turbo in our user study, and all but GPT-4 can generally response within 10 seconds. \df can sometimes be slow due to Vega-Lite rendering overhead (e.g., large datasets with more than 20,000 rows, long data threads with more than 20 charts). We envision that on-demand re-rendering of charts can improve its performance. 
\section{User Study Design}
\label{sec:evaluation}
To understand potential benefits and usability issues of \df, as well as users' interaction styles, we designed a user study that asks participants to reproduce exploratory data analysis sessions involving iteratively creating visualizations. 

\bpstart{Participants} After piloting and refining the study design with three volunteers, we recruited eight participants from a large company. Participants self-rated their skills (\autoref{fig:participants}) on a scale of 1 to 4 (``Novice,'' ``Intermediate,'' ``Proficient,'' and ``Expert'') in: (1) chart creation -- experience with chart authoring tools or libraries, (2) data transformation -- experience with data transformation tools and library expertise, (3) programming, and (4) AI assistants -- experience with large language models (e.g., ChatGPT~\cite{achiam2023gpt}) and prompting. 
\begin{figure}[t]
    \centering
    \includegraphics[width=\linewidth]{figures/participants.png}
    \caption{Participants' self-reported roles, expertise in chart creation, data transformation, programming, and AI assistants (1=novice, 4=expert), task completion time, and hints needed during study tasks.}
    \label{fig:participants}
\end{figure}

\bpstart{Setup and procedure} Each study session, conducted remotely with screen sharing, consisted of four sections within a 2-hour slot. After introduction, participants followed step-by-step instructions in the tutorial slides ($\sim$25 minutes). Participants then completed a practice task with the option to ask questions ($\sim$15 minutes) to test their understanding. Next, participants completed two study tasks, with only clarification questions allowed -- we recorded hints they requested. The two study tasks involved creating 16 visualizations, 12 requiring data transformation. Participants were encouraged to think aloud. We concluded with a debriefing to (1) compare participants' \df experiences with other tools, (2) understand their strategies using \df, and (3) gather impressions and suggestions for improvements. Breaks between phases were encouraged.

\bpstart{Tutorial and practice tasks} We used the global energy dataset (described in \autoref{sec:illustartive-scenarios}) for the tutorial and practice tasks. In the tutorial, participants followed detailed instructions to recreate the six visualizations from \autoref{fig:example-analysis-session} (all but chart \filled{4}).
In addition, participants also learned to inspect results and work with the AI's mistakes. 
In the practice tasks, participants were asked to do similar analyses but focusing on the electricity from nuclear power, they were further asked to create a bar chart to visualize the difference of energy produced from nuclear power between 2000 and 2020 for each country.

% The first task involved creating a line chart over time for CO2 emissions and then modifying this to show renewable energy trends, both fields of which were already in the dataset. 

% The next task was to create a trellis graph showing energy production broken out by three different sources (Electricity from fossil fuels, nuclear, and renewable sources). This involved restructuring the data so that a column showing the source of the energy was present as a field that could be used for the trellis chart. Subsequently, a chart that showed the percentage of renewable energy out of the total energy was created (which required another calculated field, based on the last chart that was produced). 

% The next task was to filter the previous chart by the top CO2 emission countries. Since this involved information not present in the chart created in the step directly previous but used a format based on the previous chart, there could be several ways to produce the resultant chart, either directly from the initial dataset or incrementally from the previous chart. 

%Finally, the last tasks involved the creation of an aggregated visualization of global renewable energy trends, which required going back to either the initial dataset or one of the subsequent iterations. It was important in the tutorial to offer multiple options by which people could perform the iterations so that people would subsequently have an opportunity to explore the iteration style for which they felt the most comfortable. 


%Then, in practice tasks, participants are asked to explore nuclear energy trends with four visualizations: (1) electricity form nuclear over time, (2) difference of nuclear between 2020 and 2000 (bar chart), (3) coloring differences based on increase/decrease, and (4) show only 5 countries with most nuclear increase. The first three tasks were presented with both task description and reference chart (i.e., they are chart reproduction task), but the last task only includes text description without reference, which we asked participants to verify and decide its correctness for probing their verification strategies.


\begin{figure*}[t]
    \centering
    \includegraphics[width=1\linewidth]{figures/study-tasks.png}
    \caption{The dataset and tasks in our user study. (1) Dataset 1: Understanding top earning majors and the relation between salary and women percentage. (2) Dataset 2: Exploring movie genres with best return-on-investment values (profit vs. profit ratio) and top movies. The branching directions are added for illustration; participants developed their own iteration strategies. We refer to these target charts as C1-7 for the college dataset and M1-9 for the movies dataset.}
    \label{fig:study-tasks}
\end{figure*}
\bpstart{Study tasks} To focus on participants' iterative chart creation processes, rather than their ability to create a single chart or derive insights from exploration, we used an \emph{exploration session reproduction} approach. Participants were asked to reproduce two data exploration sessions conducted by an experienced data scientist. We wanted to see if participants could iteratively create charts with \df, without requiring them to come up with exploration objectives on the fly (otherwise we would limit our participants to highly skilled data scientists). We used two exploration sessions from David Robinson's live stream analysis of Tidy Tuesday datasets.

\autoref{fig:study-tasks}-\filled{1} shows the first data exploration session: given a dataset on college majors and income data (173 rows $\times$ 7 columns), participants were asked to create seven visualizations: two basic charts and five requiring data transformation. These visualizations progressively explored the top-earning majors and the relationship between gender ratio and major salary. 
The process required participants to derive new fields (e.g., gender ratio), filter data (e.g., top 20 earning majors), derive new data (e.g., derive top earning major categories), and perform conditional formatting (e.g., color by top 4 categories and "others"). We provided a task description and reference chart (like chart reproduction studies in \cite{ren2017chartaccent,ren2018reflecting}) for all but the last two visualizations. Without reference charts for the final two, we asked participants to verify correctness, probing their verification strategies. We did not provide iteration directions, letting participants develop iteration techniques. 


\autoref{fig:study-tasks}-\filled{2} shows the second data exploration session: given a movie dataset with budget and gross information (3281 rows $\times$ 8 columns), participants created nine visualizations. These visualizations explored movies and genres with the highest return on investment, comparing profit and profit ratios. Besides two basic box plots showing budget and worldwide gross distribution, the other seven charts required data transformation, including calculation and aggregation (average profit and profit ratio for each genre), string processing (extract year for trends), filtering (year > 2000), and partitioning and ranking (top 20 movies for each metric). We hid references for the final two charts to probe participants' verification process. In the following, we use ``chart-C$k$'' and ``chart-M$k$'' to refer to the $k$-th target charts in \autoref{fig:study-tasks} for the college and movies datasets, respectively.


% \begin{figure}[t]
%     \centering
%     \includegraphics[width=\linewidth]{figures/user-study-task1.png}
%     \caption{Understanding top earning majors and the relation between salary and women percentage. Two basic charts are not shown.}
%     \label{fig:study-task1}
% \end{figure}

% \begin{figure}[t]
%     \centering
%     \includegraphics[width=\linewidth]{figures/user-study-task2.png}
%     \caption{Exploring movies genres with best return-on-investment values (profit vs. profit ratio) and top movies. Two basic charts are omitted.}
%     \label{fig:study-task2}
% \end{figure}



% \bpstart{Task 1} For the college major dataset, we asked for charts that included Median Salary by Major Category, Median Salary by Major, Median Salary by top 20 majors, and Media Salary by top 20 Majors color-coded by Major Category. We then asked about gender diversity including Median Salary by Percentage of women, the same visualization color-coded by Major Category, and color-coded by top 4 Major Categories. 

% \bpstart{Task 2} For the movie projects dataset, we asked about Production Budgets by Genre, Worldwide Gross by Genre, Median Profit by Genre, and Profit Ratios by Genre. We then asked for Median Profit by Genre over Time, subsequently narrowed to trends after 2000, and finally, Trends after the year 2000 for movies with the highest profits. 

\definecolor{myblue}{HTML}{4DABF5}
\definecolor{myyellow}{HTML}{FFCD38}


\begin{figure*}
    \centering
    \includegraphics[width=\linewidth]{figures/user-data-threads.pdf}
    \caption{Participants' workflow for study tasks in \autoref{fig:study-tasks} (C1-7 for college, M1-9 for movie). Each node represents a data table version, with \hlc[myblue!64]{blue} for initial datasets, \hlc[myyellow]{yellow} for data tables instantiating (one or multiple) target visualizations in \autoref{fig:study-tasks} (number $i$ in the node indicate the $i$-th target visualizations for the given dataset), and \hlc[gray!30]{gray} for others. Self-loop arrows indicate prompt revisions and data table updates (`$\times$2' indicates two revisions).}
    \label{fig:user-workflows}
\end{figure*}

\section{User Study Results}

Here we report user study findings including users' task completion statistics as well as their prompting, iteration and verification styles. We highlight \pquote{user quotes} and \pprompt{example prompts} in this section.

\bpstart{Task completion} All participants successfully completed all 16 visualizations (\autoref{fig:participants}): participants took less than 20 mins on average to finish the seven charts in task 1, and about 33 mins for the nine charts in task 2. Since we let participants deviate from the main exploration task (e.g., in task 2, P4 asked to sort the bar chart for top profitable movies even though it was not required), the recorded completion time is an overestimate of the actual task time. During the study, six participants asked for hints to get unstuck during tasks; we categorize them as follows:
\begin{itemize}[leftmargin=*]
\item Task clarification: P1 didn't realize that top movies were restricted to movies after 2000; P4 and P6 required hints about the difference between profit and profit ratio in task 2; P6 asked about whether the $x$-axis should be \code{Year} or \code{Date} for movie profit trends.
\item Data clarification: P6 an P8 were prompted to notice the difference between fields \code{Major} and \code{Major Category} in task 1.
\item System performance: P5 encountered a performance issue when they created large sized charts. Tn task 2, P5 created multiple bar charts with \code{Movie} mapped to the $x$-axis, resulting in bar charts containing 1300 categorical values. They were advised to reset the exploration session and resume tasks.
\item Chart encoding: P7 and P8 required hints on ``why the chart didn't render color legends'' when they didn't put a field in the color encoding; they expected to specify it only in NL input but not in the concept encoding shelf.~\footnote{In the study version, \df didn't include the feature of resolving conflicts between the users' NL and encoding shelf inputs. This feature was introduced later.}
\end{itemize}

%In general, with minimal hints, participants were able to complete the two exploration sessions themselves. 
\noindent During the debriefing, participants commented that these tasks would be much more difficult to complete with tools they are familiar with. P1, a programming expert, mentioned that they were \pquote{``obviously much faster''} with \df as it helped with data transformations. When asked about their experience comparing against chat-based AI assistants, participants noted (1) the iteration support makes it easier to create more charts and (2) the UI + NL approach in \df is more effective for communicating intent structurally. For example, P2 mentioned \pquote{``with ChatGPT, I would have to put a bit more effort to specify the instructions to get what I want, iterations here is much faster with UI.''}
%P3 \emph{``uses code interpreter separate from actual data exploration session, and transfer code to notebooks for exploration''}; 
P4 mentioned that \pquote{``with ChatGPT, you need to give much more context, I need to describe in detail about what x,y-axes should be, but here I can just provide with UI,''} and further commented that UI + NL \pquote{``helped me in framing and structuring the different transformations that we need to do to get to that end result.''}



\bpstart{Iteration styles} Participants developed their own iteration styles working with \df--\autoref{fig:user-workflows} illustrates their organization of data threads in their workspaces upon completing the study tasks. Although our participant pool of 8 did not encompass all possible users' data exploration styles with \df, we observed surprising behavior clusters and distinct approach differences. We characterize participants' iteration styles based on their preferences between ``wider'' versus ``deeper'' tree structures, ``backtrack and revise'' versus ``follow up'' for providing new instructions to the AI, as well as their preferences for including intermediate tables in their threads.

\medskip

\noindent\emph{(Wide versus deep tree organizations):} From the high-level organization of data threads, one group of participants (P1, P3, P5, P7, P8) preferred to branch out more often with shorter data threads than the other group (P2, P4, P6), who preferred to create fewer but longer data threads instead.
P1 explained that their preference of more branches with shorter data threads came from their coding style of \pquote{``creating as many as transformation as I can from one single table without generating derived tables''} to keep the system's memory usage minimal and keep the workspace \pquote{``terse.''}
On the other hand, P2, who preferred longer data threads, mentioned \pquote{``I definitely like to be able to just work on top of that and like going forward by just giving a new prompt, because it remembers the context prior to the last one. It ends up generating the right data and visualization.''} P2 further commented that \pquote{``going back created too much branching''} and they preferred to use longer threads to just provide updates for \pquote{``smooth train of thoughts.''} To effectively work with long threads, P4 organized their exploration process thoughtfully, as they were \pquote{``using the prompts as my anchor, so, when I wanted to figure out where I wanted to go, it was the prompts that I was looking for.''} 

\medskip

\noindent\emph{(Backtracking versus following-up):} We observed interesting patterns in participants' preferences when creating new charts or correcting unexpected results: some preferred revising previous instructions (evident from workflows with more self-loop arrows), while others favored following up (characterized by more forward arrows and intermediate gray data nodes). The first group, represented by P1, P2 and P3, preferred to go back and re-issue prompts, either to enrich the previous data to support multiple target visualizations (indicated by yellow nodes with multiple target charts), or to update the data to correct unexpected results. For example, when P1 and P3 worked on coloring the top 20 earning majors with their major categories (chart-C4 in \autoref{fig:study-tasks}), they revised the previous prompt (\pprompt{``show only top 20 majors based on median salary''} $\rightarrow$ \pprompt{``show only top 20 majors based on median salary, include major category''} by P1) to include \code{Major\_Category} so that both old and the new charts can be created from the same data. To correct a mistake they made in creating chart-M7 (they forgot to instruct the AI to show only top 20 movies), P3 chose to go back and revise their previous prompt (\pprompt{``calculate the profit ratio per movie (worldwide\_gross/budget) after 2000''} $\rightarrow$
\pprompt{``calculate the profit ratio per movie (worldwide\_gross/budget) after 2000 and display the top 20 higher profit ratio movies''}). P1 commented that \pquote{``I don’t like to pollute my workspace''} and \pquote{``I like to keep my workspace as clean as possible.''} P3 mentioned that their preference of revision came from the concept of building a \pquote{``global expanded dataset''} so that \pquote{``[when I] need to calculate the new thing or see a new visual I can come back to the new expanded data set.''}

On the other hand, another group, represented by P4, P5, P6, and P7, preferred not only to issue follow-up instructions for new charts but also to provide updates with very brief instructions at each step, creating many intermediate nodes along the way (gray nodes in \autoref{fig:user-workflows}). For example, P5 created chart-M7 (top movies with highest profit ratio colored by genre) in five steps: \textit{``filter movies after year 2000''} $\rightarrow$ \textit{``show top 5 highest profit ratio''} $\rightarrow$ \textit{``bring back movie''} (i.e., the {Movie} field)  $\rightarrow$ \textit{``show top 10''} $\rightarrow$ \emph{``calculate profit ratio,''} creating four intermediate nodes. %Whereas, P1 created chart-M7 with just one instruction from \textsf{movies} with \textit{``calculate the profit ratio as a ratio of worldwide gross to production budget for the top 20 movies.''} 
P5 noted that \pquote{``probably redoing would make sense, but if I can think that I can build on top of that, there is no value for me to go back and start from that, [which] kind of nullify these things [I have done],''} as they preferred to keep their work around. P6 mentioned that they adapt their iteration style based on the type of mistakes they encountered: \pquote{``if it is something intermediate where I've made the mistake, I'll go [create a new instruction] and fix the previous step''} but when it \pquote{``is a totally new kind of visualization I have in my mind''} or \pquote{``if it is something I missed altogether, I will just cancel the whole thing and start from scratch.''}

\medskip

\noindent\emph{(Choices of data to iterate on):} Participants had different strategies deciding which previous data/charts to use to create new charts. P1 chose to derive the new chart from a previous chart that shares similar visual design. For example, P1 created chart-M9 from M7 since they are both bar charts showing top ranked movies, despite one is based on profit while another is based on profit ratio. 
In a different fashion, P2, P4 and P5 often branch out based on similarity of computations used. For example, P2 created chart-M7 about movies with highest profit ratio based on chart-M6 showing profit ratio trends for each genre over time, as they shared the same computation ``profit ratio.'' 
P2 explained their data-centric approach was because they \pquote{``prefer to have more control over the data as opposed to the chart later on.''} They also appreciated that \df \pquote{``sort of brings together both data-centric and chart-centric people.''}

\bpstart{Prompt styles} Prompts created by participants are all short (less than 20 words). We observed that participants created diverse styles of prompts, both in terms of how they phrase the instruction (e.g., question, command) and the subject they asked (e.g., describing expected visual output or output data property, providing computation formula). The most common style of prompts is imperative commands, that either describe the transformation to be conducted or the property of the desired output. For example, to filter top earning movies, participants used prompts \pprompt{``show only top 20''} [P6] and\pprompt{``filter top 10 movies based on median profit''} [P5]. Participants also used command-style prompts for describing computations (e.g., \pprompt{``calculate ratio of worldwide\_gross by production\_budget''} [P5]) and for visual updates (e.g., \pprompt{``color by major category''} [P8]. 
We also observed that some participants prompted with questions (e.g., \pprompt{``can you show only the top 5 countries in terms of increases?''} [P7]), 
%, \pprompt{``What are top movies after 2000 with highest profit ratio?''} [P2]
or prompted in a chat style (e.g., \pprompt{``Good. We need to now find the Median profit ratio each year for each genre''} [P2].%, \pprompt{``I want to calculate the nuclear energy differences from 2000 and 2020 for each country''} [P3 during practice tasks]). 

One participant, P5, had a distinct prompting style, that directly asked the AI to add, mutate, or retrieve columns on top of the previous data. For example, P5 asked \pprompt{``bring back major category''} to create chart-C4 from C3, \pprompt{``divide by 100,000''} for updating profit units, \pprompt{``bring back release\_date''} before they used a follow-up command \pprompt{``show only year greater than 2,000''} to filter movies by date. P7 preferred to use more verbose prompts to reiterate the computation they intended to achieve whenever they mentioned the concept. For example, to ensure that the AI would not interpret the computation differently, they copy/pasted the formula to the prompt whenever they mentioned profit ratio --- \pprompt{``median profit ratio (worldwide\_gross/production\_budget) by year and by genre.''} P6, on the other hand, preferred to use no additional prompts and provided more descriptive field names. For example, to create chart-C5, they mapped ``\code{percentage\_of\_women\_of\_Total\_Men\_and Women}'' to $x$-axis, \code{Median\_Salary} to $y$, and provided no prompt in the input box. In fact, we observed that \df can reliably transform data with self-explanatory field names (e.g., ``\code{renewable energy percentage}'', ``\code{women percentage}'', and ``\code{difference between 2020 and 2000}'') without any additional prompts. Some participants' preference for using shorter names and additional (short) prompts was \pquote{``to minimize the error space [for AI]''} [P7]. %Participants (e.g., P4, P7) also mentioned that \textit{``when I was starting out, I wanted to make sure it is doing the right thing, but as I'm working with it for a long period of time, I think sort of the trust on the system increases''} [P4], thus their prompting style could adapt.

%\bpstart{Iteration styles}
%\df lets users develop their own iteration strategies. We observed three major distinct styles of iteration, in terms of which tables or charts participants chose to derive a new chart. 

%The first type of users preferred to achieve a particular chart through small, incremental changes from an existing chart that shared either similar data fields or similar chart configuration. For example, P2 and P3 chose to create the line chart showing profit ratio trends over time on top of the bar chart showing the average profit ratio per genre, and next visualized movies with highest profit ratio further on top, since they share the same derived field \code{profit ratio}. P2 mentioned \textit{``I definitely like to be able to just work on top of that and like going forward by just giving a new prompt, because it remembers the context prior to the last one, it ends up generating the right data and visualization.''} P2 further commented that they did not like too much branching: \textit{``...felt that it would be harder to go back to the source and fix every single time.''} P7 also preferred incremental changes, but with a focus on visual similarity as opposed to data similarity.
%Similarly, P3 didn't like to go back to earlier stages, preferring to use only the most recent iteration and continually issue new prompts to achieve the desired results.

%In contrast, the second type of users preferred to go back and re-issue a prompt to achieve all the changes from the initial data as succinctly as possible. %This was in direct contrast to the style of work that P2 and P3 preferred. 
%For example, P1 mentioned that \textit{``[I] like keeping it as terse as possible that will get me the right result.''} %P4 felt like if there was no computational linking between successive stages of iteration, then 'there's no value in going back to a previous iteration' and in fact, used iterations primarily to figure out a single prompt which would achieve the results from the original dataset.
%P4 also felt that sometimes it was more productive to just start over from the original dataset throwing out all iterations, especially when they failed to produce a desired outcome: \textit{``when we had all of those failures, I went back to the original base dataset and then frame my question there.''}
%\textit{``But when I went back to the original, so when we had all of those failures, I went back to the original base data set, and then I framed my question there, right.''}

%The third type of users primarily think about the iterations in terms for adding (or retrieving) columns from the dataset. P5 preferred to first instruct \df to add/remove columns from an existing data (e.g., bring back fields that might have been dropped in previous iterations as needed, or add a new field required for the desired chart), and then create visualization from the right data.

%Other users, (P7 for instance) used previous charts for iteration. 'I prefer to modify existing plot...' and focused on the plot rather than the data itself.

%\bpstart{Organization of iteration history}  When asked about their rationale behind branching strategies, all participants agreed that data threads are essential for managing iteration histories. Regarding their preferred organization style, P1 mentioned \textit{``I don’t like to pollute my workspace''}  and \textit{``I like to keep my workspace as clean as possible''} and thus they always chose to backtrack and fix previous instructions when encountering undesired results. P2, who mentioned \textit{``going back created too much branching''} instead preferred to follow through. P4 used prompts to help navigate iterations to find the one they were looking for: \textit{``I was using the prompts as my anchor to figure out where I wanted to go.''} P8 found it sometimes difficult to iterate in \df because data threads were \textit{``linear instead of hierarchical''}: they preferred a tree-view data thread organization, where they could scan quickly through the entire branching tree for a dataset, its transformations and visualizations and then collapse branches that were not of interest for the current goals. 

\bpstart{Verification} To proceed through iterative exploration, or repeat/correct a step, participants needed to understand the chart and verify that the transformation was performed correctly. Most of the time, participants spotted unintended output easily through incorrect patterns in rendered visualizations. This happened especially when there were differences in visual encoding (e.g., when P5 incorrectly mapped \code{release\_date} to the $x$-axis instead of \code{year} on chart-M5), cardinality (e.g., when P6 incorrectly asked the AI to color the bars by \code{major} instead of \code{major\_category} for chart-C6), or high-level patterns (e.g., when P7 requested \code{women} versus \code{median\_salary} for chart-C7, leading to results based on the count of women instead of the percentage). When the transformation is straightforward, participants visually inspected the chart and data to verify correctness. For example, after P3 asked \pprompt{``filter the year after 2000''} to show only profit ratio trends for movies after 2000 (chart-M8), they checked the $x$-axis domain and compared the generated chart with the pre-filtered one. Similarly, after P2 input \pprompt{``filter results to top 20 by major''} to find the highest earning majors (chart-C5), they referred to the previous chart with all of the majors' \code{median salary} sorted to check filtering correctness. %P2 mentioned \emph{``for simpler queries as opposed to a bit more complex queries, it was easier for me to check, in these scenarios that, hey, this is what it looks like --- it should be sorted, these are the top values that it's getting---like we went and checked the the median salary for each department thing [for chart-C3].''}

To check whether unobvious computations were done correctly (e.g., whether the LLM computed profit ratio correctly), different participants' background impacted how they validated the results: participants either referred to (1) explanations of the code, (2) the actual code (even if they are non-python programmers), or (3) values in the result table to check correctness. P3 mentioned \pquote{``as an expert, I like to see the prompt to the model, and then the code generated; but as a business user, I would imagine using more data, chart, and explanations.''} while P4 commented \pquote{``[explanation] steps were really, really helpful in terms of figuring out whether it is doing the right thing as to what I'm asking it to do. That and also the data chart underneath.''} P7 noted that, for trust, the definition of a new field is more crucial than the actual code: \pquote{``I just want to make sure that definition, like profit ratio, when I check in, I only look at those definitions if they are correct. I'm less worried about the real coding piece.''} Thus, they use code explanations frequently to check definitions. Meanwhile, P7 stated that they felt some pressure from the study environment not to spend too much time understanding code for which they were not familiar with, but they would trust code more. We also observed participants who developed trust in a workflow (by examining code and data tables) when it was straightforward, and then, they assumed the more complicated transformations built on top of these steps worked. 

%Several people commented on the desire to have the system generate code, but then to manually update that code to either achieve iterations or adjust the code to correct. P3: 'During my work, if ... I can edit the coding here because that will be like ... [the] model is giving me something.'

% P8: 'The ability to look at the Python code gave me the confidence to know it was doing the right thing, like I don't use Python, but I've been programming for many years so that the Python code looks like a lot of other code that I've seen.’

% P8 also looked at the shape/size of the resultant data: 'No, the the first thing I look at is just the general shape of the data. Like I look at the graph with the graph looks too wide, too choppy... If it's an up and down thing that you saw, that usually means... the time slice wrong in some way like my group by is off in some way, so those are the obvious ones and then the then the other ones are less obvious. So what I'm looking the first thing I do is I ballpark it and I say this is this is this even in the ballpark? '

% P7 stated that they preferred to use code rather than explanations of the code, but in the study, they used almost exclusively the explanations. They stated that they felt some pressure from the study environment not to spend too much time understanding code for which they were not familiar.

%Several people commented on the desire to have the system generate code, but then to manually update that code to either achieve iterations or adjust the code to correct. P3: 'During my work, if ... I can edit the coding here because that will be like ... [the] model is giving me something.'

\bpstart{Additional Feedback} Several users noted potential improvements of \df. P1 commented on how small interface variations might give different affordances. For instance, \pquote{``if there was a large view for data threads, it would encourage me to do more transformations and do more branching.''} P3 mentioned that they prefer the AI to ask the user to disambiguate when the intent is unclear rather than trying to solve the task with unclear specification. P7 used instructions that were very detailed and sometimes incorrect, which in turn, made iteration more difficult, since it was difficult to incrementally modify these instructions. We discussed the potential of having templates or AI feedback for instruction crafting to reduce errors.

\section{Discussion and Future Work}

\bpstart{Supporting recommendations in exploratory analysis} 
\df focuses on visualization authoring, where an AI completes tasks needed to achieve a user's intended action. We envision that \df can be enhanced with recommendation capabilities like Voyager~\cite{Wongsuphasawat2017voyager2}, Draco~\cite{moritz2018formalizing}, and Lux~\cite{lee2021lux} for suggesting visualization goals to help users ``cold start'' their analysis. \df's designs can benefit user experiences with visualization recommendation tools in two ways. First, because \df supports visualization beyond initial data formats, it overcomes the limitation of most existing tools, which only consider fields in the input table for recommendation. Second, \df's data threads provide a natural way for users to follow up the system's initial recommendations, either to dive deeper into an exploration direction, revising suggested charts, or to ask for different recommendations. To achieve this, we can add a recommendation component that can suggest a list of fields to be explored and let \df prepare data to surface the fields and create visualizations. The initial recommendation of the fields of interests can be generated either automatically from the analysis of input data characteristics or in a mixed-initiative approach, i.e., leveraging AI to generate them using a high-level natural language instruction provided by the user --- these fields do not have to be fields in input data, as \df can transform the data to derive them from existing ones. While \df's data transformation ability can extend the visualization exploration space, thus bringing in more potential insights to be discovered, it also increases the chances of suggesting field combinations that are either trivial, distracting, or even biased. Therefore, as part of future work, it would be valuable to explore ways to support visual recommendation in a larger exploration space, especially managing and communicating exploration paths to the user to prevent unintentional bias towards an undesired direction.


\bpstart{Coordinating data transformation and chart editing} \df derives new data based on users' inputs to instantiate the chart design, but it does not modify the chart itself (represented as a Vega-Lite specification). When the user wants to refine the chart design (e.g., updating color scheme or $x$-axis ordering), they edit it through GUI widgets in the encoding panel after the chart is created. This design leverages the natural and precise nature of UI updates, providing immediate visual feedback~\cite{vaithilingam2024dynavis}. It also utilizes current models' strengths in data transformation for more reliable outputs~\cite{gao2023text,lai2023ds}. In contrast, current models perform less effectively in editing charts or generating charts from data based on NL instructions, even when the data is prepared~\cite{chen2024viseval}. Despite this, some participants in the user study showed interest in asking \df to perform chart edits within the chart builder alongside data transformation. A potential solution is an agent-based system~\cite{wu2023autogen,zhang2024training} that plans whether to transform data, edit the chart script, or both based on user inputs, and dispatches agents to handle these tasks. The key challenge is managing response time and maintaining reliability, as AI agents often require multiple interactions to reach consensus. %A potential direction is to experiment whether a multi-agent system composed of small language models~\cite{abdin2024phi} (for reduced model inference costs) can outperform a single LLM.  %It would be interesting future work to explore ways to integrate these approaches into interactive systems like \df to increase system expressiveness, while maintaining reliability and efficiency (e.g., minimizing chances for the planner to route user requests to wrong AI agent and reducing overall agent interaction rounds). 

%Furthermore, as \df currently focuses on grammar of graphics-based visualization supported in Vega-Lite, it provides limited supports of custom chart designs (e.g., new layouts, interaction, animation, or annotation) that require extensive editing after the chart is created. To support advanced editing, there are opportunities to incorporate direct manipulation interfaces in \df (e.g., canvas UI) so that users can directly manipulate marks~\cite{ren2019charticulator}, visual objects, and layouts~\cite{tsandilas2020structgraphics,saket2016visualization} to revise chart and improve its aesthetics. 

\bpstart{Asking users to clarify ambiguous inputs} \df adopts a generation verification approach: AI attempts to complete the user's request, and the user inspects the result to provide follow-up instructions. This interaction loop is enhanced by \df's local data thread design. There is an opportunity to make AI more proactive, such as actively seeking clarification from users when their inputs are ambiguous, before attempting to solve the task. This could reduce users' verification and revision efforts. For example, when the user issues an unclear request (e.g., \textit{``show top 5''}), the system can first analyze the goal, and then either present a refined goal for confirmation (e.g., \textit{``do you mean top 5 by renewable percentage?''}) or ask the users to clarify their intent (e.g., \textit{``what criteria should be used for ranking?''}). This proactive approach could also promote users' trust in the AI system. It is an interesting research direction to explore ways to prompt or train AI models to ask only necessary clarification questions, preventing users from being overwhelmed with low-level questions that might interrupt their workflow.

\bpstart{Study limitations} In our user study, we used the reproduction of professional data analysts' exploration sessions as the study tasks, rather than asking participants to perform free explorations. This choice was made to minimize the impact of participants' data analysis skills on their experience with \df, as our goal was not to assess their exploration skills. A follow-up study, where participants perform open exploration with their own data, can further investigate how \df can assist analysts with planning during exploration. In addition, as a limitation of our lab study, we could not capture users' longer-term learning effects. A future longitudinal study could further investigate how users' expectations with \df change over time and how this affects their specification styles and iteration strategies. %Therefore, it would be helpful to conduct a longitudinal study, asking participants to use \df in their workflow to explore data of their interests, characterizing their exploration processes and final conclusions using \df. 
\section{Related Work}
\label{related-work}

Compared to its predecessor, Data Formulator~\cite{wang2023data}, \df has transitioned from a single-turn chart authoring tool into an iterative visualization tool designed for data exploration. Concretely, Data Formulator~\cite{wang2023data} is a single-turn authoring tool that leverages different authoring paradigms for various types of data transformations. It uses programming-by-example for table reshaping and employs LLMs to generate code for single column derivation. However, users may struggle with choosing the appropriate paradigm for the required transformations. \df unifies the interaction paradigms with a blended UI and natural language input design, supporting iterative authoring. This allows users to build new charts from previous ones with minimal additional specification. \df's new interaction approaches not only broaden the expressiveness of supported data transformations but also reduce the users' specification overhead. We next illustrate related work on  chart authoring, data transformation, and data exploration tools that inspired the design of \df.

\bpstart{LLM-powered visualization tools} Large language models' code generation ability~\cite{achiam2023gpt,chen2021evaluating,lozhkov2024starcoder,touvron2023llama} motivates the designs of new AI-powered visualizations tools~\cite{dibia2019data2vis,maddigan2023chat2vis,tian2024chartgpt,wang2023data} that allows users to create visualization using high-level natural language descriptions. For example,  LIDA~\cite{dibia2023lida} can summarize data and use LLM to generate python code to generate visualizations. Because LLMs can struggle in understanding complex chart logic,  ChartGPT~\cite{tian2024chartgpt} decomposes visualization tasks into fine-grained reasoning pipelines (e.g., column selection, filtering, chart type selection, visual encoding), using chain-of-thoughts prompting~\cite{wei2022chain}. As single-turn interactive tools, they are not suitable for iterative analysis. For multi-turn interactions, users can directly chat with LLMs in Code Interpreter~\cite{achiam2023gpt} or Chat2Vis~\cite{maddigan2023chat2vis}. Code Interpreter equips the LLM with a Python interpreter so that the model can generate and execute code to transform data and create charts; Chat2Vis includes visualization-specific prompts to help the model generate visualizations more reliably. Since these tools organize the dialog linearly, users need to put in extra effort to clarify the context when there are branches, to reduce the chances of models applying incorrect contexts and making mistakes in the new task~\cite{liu2024lost,zhang2023tell,hsieh2024ruler}. %Since these tools are based on NL inputs, users have to convert designs in their mind into potentially verbose texts to communicate their intent.

\df is also an LLM-powered tool that shares similar prompt designs to LIDA and Chat2Vis (e.g., the use of data summaries) and supports NL interaction. The key difference is that \df blends UI and natural language inputs for chart specification, balancing precision and flexibility, rather than requiring users to describe everything in text. 
\df's data threads generalize linear contexts used in existing dialog systems, allowing users to control the iteration direction by providing authoring contexts to the AI model.

\bpstart{Other AI and synthesis-powered tools} Besides LLM-powered tools above, neural semantic parsing~\cite{mitra2022facilitating,narechania2020nl4dv,chen2022type}, and program synthesis-based tools~\cite{wang2021falx} have also been developed to address the visualization challenge. For example, NL4DV~\cite{narechania2020nl4dv} and NcNet~\cite{luo2021natural} leverage recurrent neural networks trained to translate NL queries into charts. NL2Vis~\cite{wu2022nl2viz} and Graphy~\cite{chen2022type} use a semantic parser to extract entities from the user's NL query and apply program synthesis algorithms to compose charts. Unlike LLMs, these tools are more restrictive in the supported data transformations and chart types, requiring very specific chart descriptions from the user.  While programming-by-examples (PBE) techniques are developed to tackle data reshaping challenges in chart authoring (e.g., Falx~\cite{wang2021falx} and Data Formulator~\cite{wang2023data}'s reshaping module), users need to prepare low-level examples to demonstrate the transformation intent, which deviates users from the high-level visualization workflow. 
%Unlike LLM-based tools where the user can directly have conversation with the model to disambiguate inputs, semantic parsing and PBE-based tools develop special techniques for resolving ambiguous user intent. 
For disambiguation, DataTone~\cite{DBLP:conf/uist/GaoDALK15} introduces disambiguation widgets for users to experiment with different entity extraction outputs for the generated query, and users can inspect paraphrased queries (in NL) to resolve ambiguity; Falx~\cite{wang2021falx} previews charts from multiple versions of data consistent with user examples. Benefiting from the use of LLMs, \df is more expressive. Inspired by how prior work displays candidate results and explains code to help users understand system outputs~\cite{DBLP:conf/uist/GaoDALK15,gu2023analysts,wang2023data}, \df displays generated code, data, chart and code explanation to assist user inspection. %To resolve ambiguous outputs, the user can use data threads to follow up or backtrack and revise the their instructions.

%LIDA abstract: We present LIDA, a novel tool for generating grammar-agnostic visualizations and infographics. LIDA comprises of 4 modules - A SUMMARIZER that converts data into a rich but compact natural language summary, a GOAL EXPLORER that enumerates visualization goals given the data, a VISGENERATOR that generates, refines, executes and filters visualization code and an INFOGRAPHER module that yields data-faithful stylized graphics using IGMs.

% DATATONE abstract: In this work we propose a mixed-initiative approach to managing ambiguity in natural language interfaces for data visualization. We model ambiguity throughout the process of turning a natural language query into a visualization and use algorithmic disambiguation coupled with interactive ambiguity widgets.

%NL2Vis tools such as Lida~\cite{dibia2019data2vis} and NL4DV~\cite{narechania2020nl4dv} take a high-level description as input from the user and generate visualizations as output. Despite these tools excel at basic visualization tasks, they fail to produce high-quality results if the user description is insufficient -- especially when the task is too complex to be described in a short sentence. As the result, these tools are not suitable for iterative analysis, since the users need to describe their intent from scratch every iteration and the analysis goal can be difficult to describe in full in one sentence.

%Chat-based, or multi-turn NL, tools (...) are then developed to support the iterative visualization, where the user can provide follow-up instructions to ask the system to refine previous results so that the users don't need to start from scratch. Because these tools store the interaction history in a linear context, the user needs to put extra efforts in formulating a prompt to elaborate the contexts of the refinement instructions, otherwise the system could produce undesired results. Furthermore, because these systems expect only NL inputs, the user has to describe their chart configuration in NL, even though it can be more easily achieved using simple UI interactions as demonstrated in~\df. %, which adds additional specification efforts from the user.

%To remedy the specification challenge, tools like Data Formulator~\cite{wang2023data} provide multi-modal user interfaces that allows users to choose between UI and NL for different types tasks. However, it was shown

%Furthermore, DataTone~\cite{DBLP:conf/uist/GaoDALK15} like NL4DV~\cite{narechania2020nl4dv}, Lida~\cite{dibia2019data2vis}, DataTone~\cite{DBLP:conf/uist/GaoDALK15}, Data Formulator~\cite{wang2023data}, Data2Vis~\cite{dibia2019data2vis}, and Chat2Vis~\cite{maddigan2023chat2vis} are designed to let users create visualization from high-level inputs like natural languages, examples and demonstrations. 

%Data Anvil shares some similar designs with these tools. For example, Data Anvil leverages an NL interface to guide data derivation; Data Anvil's prompting strategy to interact with AI follows the similar designs in Lida and Data Formulator, using chain-of-thoughts prompting~\cite{wei2022chain} and data summary. There are several key differences. First, Data Anvil considers data transformation as part of the visualization authoring process~\cite{ren2018reflecting}, as opposed to focus only on chart generation from tidy inputs~\cite{narechania2020nl4dv}. Second, Data Anvil centers its designs around iteration, inspired by computation notebook~\cite{deline2021glinda,rule2018exploration}, dialog systems~\cite{gao2019neural,huang2020challenges} and exploratory search~\cite{gao2023neural}: comparing to tools that expect users to specify their intent all at once, Data Anvil let users break down complex tasks to solve them incrementally. Last, Data Anvil combines UI and NL for chart specification, inspired by multi-modal interfaces from other domains~\cite{DBLP:conf/uist/LiRJSMM19pumice}, so that user would not need verbose inputs for complex tasks needed in chat-based tools~\cite{maddigan2023chat2vis}. These designs better align Data Anvil for iterative visualization tasks.
%AI-assistants in computation notebooks~\cite{mcnutt2023design} and code editors~\cite{barke2023grounded,nguyen2022empirical} are also designed for data analysis. Despite their flexibility, they require users to be proficient at programming~\cite{barke2023grounded,gu2023analysts}.

%Since AI systems can generate incorrect results, verification techniques are developed to help user understand and correct errors. Data Anvil presents chart, data and code~\cite{wang2021falx,wang2023data} to  to assist users to examine both the computation and produced artifacts~\cite{gu2023analysts}. Data Anvil also incorporates AI-generated explanations to help non-coder understand the process~\cite{dibia2023lida}. Data Anvil then allows users to iterate with AI to correct errors. In future, Data Anvil can also benefit from explaining the computation process as diagrams~\cite{xiong2022visualizing,jiang2023log} or supporting interactive code or data editing~\cite{gordon2023co,drosos2023fxd} to improve users' trusts.


\bpstart{Visualization grammars and tools} The grammar of graphics~\cite{DBLP:books/daglib/0024564} inspired many modern visualization grammars (e.g., ggplot2~\cite{wickham2009ggplot2}, Vega-Lite~\cite{satyanarayan2017vegalite},
Altair~\cite{vanderplas2018altair}), where visualizations are mainly described by mappings from data columns to visual channels. Comparing to more expressive languages like D3~\cite{bostock2011d3} and Atlas~\cite{liu2021atlas}, high-level grammars hide the computation process of linking data items to visual objects to reduce visualization effort. Powered by these high-level grammars, interactive tools like Lyra~\cite{satyanarayan2014lyra}, Data Illustrator~\cite{liu2018data}, Charticulator~\cite{ren2019charticulator}, Tableau~\cite{stolte2002query}) have been introduced, where users leverage the shelf-configuration interface to specify visual encodings. To reduce authoring efforts, tools like Voyager~\cite{Wongsuphasawat2017voyager2}, Lux~\cite{lee2021lux}, and Draco~\cite{moritz2018formalizing} leverage rule and logic-based recommendation techniques to suggest visualizations from users' partial chart specifications. For example, Voyager lets users put a wildcard field into an encoding slot, and then automatically instantiates the wildcard field with different existing fields from the table, to produce interesting charts for users to explore. Note that these tools all require tidy input data~\cite{wickham2014tidy-data}, where all fields to be visualized should be data columns. Thus, users need to learn to use data transformation tools to prepare data~\cite{the_pandas_development_team_2023_7741580,wickham2019tidyverse,raman2001potter,kandel2011wrangler,polozov2015flashmeta,jin2017foofah,beth2020mage,huang2023interactive,chen2020multi}.

\df benefits from Vega-Lite's expressiveness to support rich visualization designs. \df inherits the shelf-configuration design from existing interactive tools and enhanced it with NL inputs for users to create charts that require data transformation.
While \df's custom fields resemble wildcard fields in Voyager~\cite{Wongsuphasawat2017voyager2}, they are semantically different: a custom field is for a field that users desire to visualize but not yet exist in the current table, requiring data transformation to surface, while a wildcard field refers to a field in the current table that the user does not specify explicitly. There is potential to unify these two as ``wildcard custom fields'' so that the system can recommend unspecified fields beyond the available fields in the current data (leveraging data transformation), which would broaden the exploration space.

%Programming libraries like pandas~\cite{the_pandas_development_team_2023_7741580}, R tidyverse~\cite{wickham2019tidyverse}, SQL~\cite{date1989guide}, and tools like Potter's Wheel~\cite{raman2001potter}, Wrangler~\cite{kandel2011wrangler}, Tableau Prep are developed to support data transformation. These tools support expressive data transformation idioms like relational algebra, reshaping, window function and string processing that are essential for data analysis. To lower the skill requirement of data transformation, automated data transformation tools are proposed, including programming-by-example tools  (e.g., FlashFill~\cite{polozov2015flashmeta,gulwani2017program}, Scythe~\cite{wang2017synthesizing}, Foofah~\cite{jin2017foofah}), mixed-initiative tools (e.g., Mage~\cite{beth2020mage}, Wrangler~\cite{kandel2011wrangler}), and natural language tools (e.g., NL2SQL~\cite{yu2018syntaxsqlnet,chen2020multi}, Rigel~\cite{huang2023interactive}). 
%Despite their convenience, visualization authors need conceptual knowledge to understand the desired data format. %Furthermore, many of existing tools restrict language expressiveness using domain specific  to improve system performance and reduce ambiguities from user specification, which limits their applicability to the authoring of rich visualizations.
%Data Anvil inherits the shelf-configuration design from existing interactive tools for chart specification. Data Anvil leverages LLMs~\cite{chen2021evaluating,achiam2023gpt} to generate expressive code to automate data transformation. In particular, Data Anvil unifies data and chart in the same contexts to reduce user efforts.

\bpstart{Exploration history} Graphical history~\cite{heer2008graphical} and data provenance~\cite{buneman2001and} are essential in visualization authoring, especially in exploration tasks where branching and iterations are common. In computation notebooks, the exploration history is organized based on code blocks~\cite{mcnutt2023design,observable}. Data transformation tools like somnus~\cite{xiong2022visualizing} and Tableau Prep visualize data provenance based on transformation operators. Directed-graph models~\cite{shi2019task,kim2017graphscape} based on visual similarity are also used for visualization organization. To assist data scientists manage (messy) programming histories in computation notebooks, Verdant~\cite{kery2017variolite} introduces a design that visualizes users' edit histories of notebook and artifacts, allowing them to revisit different versions of the notebook; code gathering tools~\cite{head2019managing} leverage data dependency to extract a clean and minimal code snippet from a notebook that can reproduce a variable of interest. To support the management of different versions of code snippets created during the ideation process, Variolite~\cite{kery2017variolite} allows users to explicitly create branches when experimenting different implementations of a function and to switch among them later on. 

\df's data threads draw inspiration from these systems. The key difference is that data threads are designed for users to steer iteration directions by providing authoring contexts with AI. This approach organizes history around high-level user interactions with AI and hides operator-level details. We characterized users' interaction strategies based on their exploration tree~\cite{white2007investigating}. Provenance management techniques for notebooks can be applied to manage long data threads users created across different sessions (e.g., compressing long data threads into shorter ones with summaries). In the future, \df could render data threads as hierarchical trees~\cite{kim2017graphscape} to support navigation of large data threads in multiple granularities. Additionally, it could incorporate version toggles, similar to Variolite, allowing users to explore different versions of generated code more compactly, rather than presenting all exploration branches as separate data threads.

%AI-assistants in computation notebooks~\cite{mcnutt2023design} and code editors~\cite{barke2023grounded,nguyen2022empirical} are also designed for data analysis. Despite their flexibility, they require users to be proficient at programming~\cite{barke2023grounded,gu2023analysts}.

\bpstart{Multi-modal interaction} Despite natural language providing flexible and expressive interactions between human and AI, NL-only interaction is not always optimal for the users to clearly convey their intent, especially for conveying designs  pictured only in the user's  mind. To address this limitation, multi-modal models like ChatGPT~\cite{achiam2023gpt} and Gemini~\cite{reid2024gemini} have been introduced, allowing users to provide audios and images in their conversation with AI. New interactive tools are also developed to support multi-modal interaction. For example, DirectGPT~\cite{masson2024directgpt} allows users to directly point and click on a canvas to specify contexts or objects that NL instruction is based on to reduce prompting effort, Mage~\cite{beth2020mage} provides interactive widgets for users to control content in notebook, and DynaVis~\cite{vaithilingam2024dynavis} generates UI widgets dynamically based on user's NL inputs for chart editing with LLMs so that users can explore and repeat edits and see instant visual feedback from edits. \df's chart builder bridges the precision and affordance of GUI interaction with flexibility of NL inputs and thus exploits a multi-modal UI design for visualization authoring.


\section{Conclusion}
\label{sec:conclusion}

Visualization authors often create visualizations iteratively, alternating between data transformation and visualization steps. This process requires proficiency with tools and considerable effort to manage various versions of data and charts. 
Although AI-powered tools aim to reduce user effort, they fall short for iterative analysis, expecting users to specify their intent at once with NL inputs.
We present \df, an interactive system for iterative visualization authoring. \df features a multi-modal UI that allows users to specify visualizations using a blend of UI and NL inputs, enabling users to convey complex designs more precisely without verbose prompts.
To help user manage iteration directions, \df introduces data threads for users to navigate, branch, and reuse previous designs. In a user study with eight participants reproducing two challenging data exploration sessions consisting of 16 visualizations, we observed that \df enabled participants to develop their own iteration and verification strategies confidently with minimal hints.

% \begin{acks}
% The second author was supported by the Institute of Information and Communications Technology Planning and Evaluation (IITP) Grant funded by the Korean Government (MSIT), Artificial Intelligence Graduate School Program, Yonsei University, under Grant RS-2020-II201361.
% \end{acks}

%\section{Supplementary Materials}

Please check out our supplementary materials for (1) \df introduction video,  (2) unedited videos demonstrating \df's experiences to complete tasks in \autoref{sec:illustartive-scenarios} and user study tasks~\autoref{sec:evaluation}, and (3) user study handout.

\begin{comment}
\section{System Implementation Details}

\bpstart{Currently supported chart types}  \df  supports 16  charts types across five categories (scatter, bar, line statistical, custom). We listed out the Vega-Lite mark(s) used to compose the chart as well as visual channels provided to the user. 

\medskip

\begin{center}
\small
    \begin{tabular}{|c|c|l|}
       \hline Chart Type  & Vega-Lite Mark(s) & Visual channels \\\hline
        Scatter Plot & circle & \textsf{x}, \textsf{y}, \textsf{color}, \textsf{column}, \textsf{row} \\
        Ranged Dot Plot & circle + line & \textsf{x}, \textsf{y}, \textsf{color}\\ \hdashline
        
        Bar Chart & bar & \textsf{x}, \textsf{y}, \textsf{color}, \textsf{column}, \textsf{row} \\
        Grouped Bar Chart & bar & \textsf{x}, \textsf{y}, \textsf{group} \\
        Stacked Bar Chart & bar & \textsf{x}, \textsf{y}, \textsf{color}, \textsf{column}, \textsf{row} \\
        \hdashline
        Line Chart & line & \textsf{x}, \textsf{y}, \textsf{color}, \textsf{column}, \textsf{row} \\
        Dotted Line Chart & circle + line & \textsf{x}, \textsf{y}, \textsf{color}, \textsf{column}, \textsf{row} \\
        Heat Map & rect & \textsf{x}, \textsf{y}, \textsf{color}, \textsf{column}, \textsf{row} \\
        \hdashline
        Linear Regression & circle + line & \textsf{x}, \textsf{y}, \textsf{color}, \textsf{column}, \textsf{row}\\
        Histogram & bar & \textsf{x}, \textsf{color}, \textsf{column}, \textsf{row} \\
        {Boxplot} & boxplot &  \textsf{x}, \textsf{y}, \textsf{color}, \textsf{opacity}, \textsf{column}, \textsf{row} \\
        \hdashline
        Custom Point & circle & \textsf{x}, \textsf{y}, \textsf{color}, \textsf{size}, \textsf{shape}, \textsf{opacity}, \textsf{column}, \textsf{row} \\
        Custom Line & line & \textsf{x}, \textsf{y}, \textsf{detail}, \textsf{color}, \textsf{opacity}, \textsf{column}, \textsf{row} \\
        Custom Bar & bar & \textsf{x}, \textsf{y}, \textsf{color}, \textsf{size}, \textsf{shape}, \textsf{opacity}, \textsf{column}, \textsf{row} \\
        Custom Rect & rect & \textsf{x}, \textsf{y}, \textsf{x2}, \textsf{y2}, \textsf{color}, \textsf{size}, \textsf{opacity}, \textsf{column}, \textsf{row} \\
         Custom Area & area & \textsf{x}, \textsf{y}, \textsf{x2}, \textsf{y2}, \textsf{color}, \textsf{opacity}, \textsf{column}, \textsf{row} \\\hline
    \end{tabular}
\end{center}

\medskip

\df includes chart types with overlapping expressiveness (e.g., ``stacked bar chart'' can also be composed from ``custom bar'') so that novice users won't be overwhelmed with unfamiliar visual channels when working with basic charts. \df includes a glyph icon for each chart to assist user navigation. If a developer would like to extend \df to support new chart types, it can be achieved easily by providing a Vega-Lite template along with rules about how visual encodings from the user will be routed to different slots in the template. 

\bpstart{The system prompt} We provide the system prompt used by \df to communicate with the LLM. The system prompt includes role definition and few-shot examples. New user inputs are appended as \code{[CONTEXT]} and \code{[GOAL]} to end the system prompt, and the LLM is asked to complete the \code{[OUTPUT]} section.

\definecolor{LightGray}{gray}{0.95}
\begin{minted}[
baselinestretch=1.2,
breakanywhere,
bgcolor=LightGray,
fontsize=\footnotesize,
]{text}
You are a data scientist to help user to transform data that will be used for visualization.
The user will provide you information about what data would be needed, and your job is to create a python function based on the input data summary, transformation instruction and expected fields.
The users' instruction includes "expected fields" that the user want for visualization, and natural langauge instructions "goal" that describe what data is needed.

Concretely, you should first refine users' goal and then create a python function in the [OUTPUT] section based off the [CONTEXT] and [GOAL]:

    1. First, refine users' [GOAL]. The main objective in this step is to check if "visualization_fields" provided by the user are sufficient to achieve their "goal". Concretely:
        (1) based on the user's "goal", elaborate the goal into a "detailed_instruction".
        (2) determine "output_fields", the desired fields that the output data should have to achieve the user's goal, it's a good idea to include intermediate fields here.
        (2) now, determine whether the user has provided sufficient fields in "visualization_fields" that are needed to achieve their goal:
            - if the user's "visualization_fields" are sufficient, simply copy it.
            - if the user didn't provide sufficient fields in "visualization_fields", add missing fields in "visualization_fields" (ordered them based on whether the field will be used in x,y axes or legends);
                - "visualization_fields" should only include fields that will be visualized (do not include other intermediate fields from "output_fields")  
                - when adding new fields to "visualization_fields", be efficient and add only a minimal number of fields that are needed to achive the user's goal. generally, the total number of fields in "visualization_fields" should be no more than 3 for x,y,legend.

    Prepare the result in the following json format:

```
{
    "detailed_instruction": "..." // string, elaborate user instruction with details if the user
    "output_fields": [...] // string[], describe the desired output fields that the output data should have based on the user's goal, it's a good idea to preserve intermediate fields here (i.e., the goal of transformed data)
    "visualization_fields": [] // string[]: a subset of fields from "output_fields" that will be visualized, ordered based on if the field will be used in x,y axes or legends, do not include other intermediate fields from "output_fields".
    "reason": "..." // string, explain why this refinement is made
}
```

    2. Then, write a python function based on the refined goal, the function input is a dataframe "df" and the output is the transformed dataframe "transformed_df". "transformed_df" should contain all "output_fields" from the refined goal.
The python function must follow the template provided in [TEMPLATE], do not import any other libraries or modify function name. The function should be as simple as possible and easily readable.
If there is no data transformation needed based on "output_fields", the transformation function can simply "return df".

[TEMPLATE]

```python
import pandas as pd
import collections
import numpy as np

def transform_data(df):
    # complete the template here
    return transformed_df
```

    3. The [OUTPUT] must only contain a json object representing the refined goal (including "detailed_instruction", "output_fields", "visualization_fields" and "reason") and a python code block representing the transformation code, do not add any extra text explanation.

For example:

[CONTEXT]

Here are our datasets, here are their field summaries and samples:

table_0 (us_covid_cases) fields:
	Date -- type: object, values: 1/1/2021, 1/1/2022, 1/1/2023, ..., 9/8/2022, 9/9/2020, 9/9/2021, 9/9/2022
	Cases -- type: int64, values: -23999, -14195, -6940, ..., 1018935, 1032159, 1178403, 1433977

table_0 (us_covid_cases) sample:
```
|Date|Cases
0|1/21/2020|1
1|1/22/2020|0
2|1/23/2020|0
3|1/24/2020|1
4|1/25/2020|1
......
```

[GOAL]

{
    "instruction": "calculate 7-day moving average",
    "visualization_fields": ["Date", "7-day average cases"]
}

[OUTPUT]

{  
    "detailed_instruction": "Calculate the 7-day moving average of COVID-19 cases over time.",  
    "output_fields": ["Date", "Cases", "7-day average cases"],  
    "visualization_fields": ["Date", "7-day average cases"],  
    "reason": "To calculate the 7-day moving average, the 'Cases' field is required, but it is not needed for visualization. The provided fields are sufficient to achieve the goal."  
}  

```python
import pandas as pd  
import collections  
import numpy as np  
  
def transform_data(df):  
    # Convert Date column to datetime  
    df['Date'] = pd.to_datetime(df['Date'])  
      
    # Sort the dataframe by Date  
    df = df.sort_values('Date')  
      
    # Calculate the 7-day moving average of cases  
    df['7-day average cases'] = df['Cases'].rolling(window=7).mean()  
      
    # Select the output fields  
    transformed_df = df[['Date', 'Cases', '7-day average cases']]  
      
    return transformed_df  
```
\end{minted}


% [CONTEXT]

% Here are our datasets, here are their field summaries and samples:

% table_0 (weather_seattle_atlanta) fields:
% 	Date -- type: object, values: 1/1/2020, 1/10/2020, 1/11/2020, ..., 9/6/2020, 9/7/2020, 9/8/2020, 9/9/2020
% 	City -- type: object, values: Atlanta, Seattle
% 	Temperature -- type: int64, values: 30, 31, 32, ..., 83, 84, 85, 86

% table_0 (weather_seattle_atlanta) sample:
% ```
% |Date|City|Temperature
% 0|1/1/2020|Seattle|51
% 1|1/1/2020|Atlanta|45
% 2|1/2/2020|Seattle|45
% 3|1/2/2020|Atlanta|47
% 4|1/3/2020|Seattle|48
% ......
% ```

% [GOAL]

% {
%     "instruction": "create a scatter plot to with seattle and atlanta temperatures on x,y axes, color points by which city is warmer",
%     "visualization_fields": []
% }

% [OUTPUT]

% {  
%     "detailed_instruction": "Create a scatter plot to compare Seattle and Atlanta temperatures with Seattle temperatures on the x-axis and Atlanta temperatures on the y-axis. Color the points by which city is warmer.",  
%     "output_fields": ["Date", "Seattle Temperature", "Atlanta Temperature", "Warmer City"],  
%     "visualization_fields": ["Seattle Temperature", "Atlanta Temperature", "Warmer City"],  
%     "reason": "To compare Seattle and Atlanta temperatures with Seattle temperatures on the x-axis and Atlanta temperatures on the y-axis, and color points by which city is warmer, separate temperature fields for Seattle and Atlanta are required. Additionally, a new field 'Warmer City' is needed to indicate which city is warmer."  
% }  

% ```python
% import pandas as pd  
% import collections  
% import numpy as np  
  
% def transform_data(df):  
%     # Pivot the dataframe to have separate columns for Seattle and Atlanta temperatures  
%     df_pivot = df.pivot(index='Date', columns='City', values='Temperature').reset_index()  
%     df_pivot.columns = ['Date', 'Atlanta Temperature', 'Seattle Temperature']  
      
%     # Determine which city is warmer for each date  
%     df_pivot['Warmer City'] = df_pivot.apply(lambda row: 'Atlanta' if row['Atlanta Temperature'] > row['Seattle Temperature'] else 'Seattle', axis=1)  
      
%     # Select the output fields  
%     transformed_df = df_pivot[['Date', 'Seattle Temperature', 'Atlanta Temperature', 'Warmer City']]  
      
%     return transformed_df 
%```

\end{comment}

% \section*{Supplemental Materials}
% \label{sec:supplemental_materials}

% We include the user study tutorials and tasks in our supplementary materials. Furthermore, we include the following unedited videos demonstrating \tool's experiences:
% \begin{itemize}[leftmargin=*]\itemsep-3pt
% \item Exploration of renewable energy trends as shown in \autoref{sec:illustartive-scenarios}.
% \item Exploring nuclear energy differences for each country between 2020 and 2000 from the global energy data (user study practice task).
% \item Exploring the relation between college major, salary and women percentage (user study task 1~\autoref{fig:study-task1}).
% \item Exploring most profitable genres and movies, comparing both profit and profit ratio as metrics (user study task 2~\autoref{fig:study-task2}).
% \item Demonstration of the ``smart sort'' feature to sort months.
% \end{itemize}

\bibliographystyle{ACM-Reference-Format}
\bibliography{main}


% \appendix % You can use the `hideappendix` class option to skip everything after \appendix

% \section{Supplementary Materials}

Please check out our supplementary materials for (1) \df introduction video,  (2) unedited videos demonstrating \df's experiences to complete tasks in \autoref{sec:illustartive-scenarios} and user study tasks~\autoref{sec:evaluation}, and (3) user study handout.

\begin{comment}
\section{System Implementation Details}

\bpstart{Currently supported chart types}  \df  supports 16  charts types across five categories (scatter, bar, line statistical, custom). We listed out the Vega-Lite mark(s) used to compose the chart as well as visual channels provided to the user. 

\medskip

\begin{center}
\small
    \begin{tabular}{|c|c|l|}
       \hline Chart Type  & Vega-Lite Mark(s) & Visual channels \\\hline
        Scatter Plot & circle & \textsf{x}, \textsf{y}, \textsf{color}, \textsf{column}, \textsf{row} \\
        Ranged Dot Plot & circle + line & \textsf{x}, \textsf{y}, \textsf{color}\\ \hdashline
        
        Bar Chart & bar & \textsf{x}, \textsf{y}, \textsf{color}, \textsf{column}, \textsf{row} \\
        Grouped Bar Chart & bar & \textsf{x}, \textsf{y}, \textsf{group} \\
        Stacked Bar Chart & bar & \textsf{x}, \textsf{y}, \textsf{color}, \textsf{column}, \textsf{row} \\
        \hdashline
        Line Chart & line & \textsf{x}, \textsf{y}, \textsf{color}, \textsf{column}, \textsf{row} \\
        Dotted Line Chart & circle + line & \textsf{x}, \textsf{y}, \textsf{color}, \textsf{column}, \textsf{row} \\
        Heat Map & rect & \textsf{x}, \textsf{y}, \textsf{color}, \textsf{column}, \textsf{row} \\
        \hdashline
        Linear Regression & circle + line & \textsf{x}, \textsf{y}, \textsf{color}, \textsf{column}, \textsf{row}\\
        Histogram & bar & \textsf{x}, \textsf{color}, \textsf{column}, \textsf{row} \\
        {Boxplot} & boxplot &  \textsf{x}, \textsf{y}, \textsf{color}, \textsf{opacity}, \textsf{column}, \textsf{row} \\
        \hdashline
        Custom Point & circle & \textsf{x}, \textsf{y}, \textsf{color}, \textsf{size}, \textsf{shape}, \textsf{opacity}, \textsf{column}, \textsf{row} \\
        Custom Line & line & \textsf{x}, \textsf{y}, \textsf{detail}, \textsf{color}, \textsf{opacity}, \textsf{column}, \textsf{row} \\
        Custom Bar & bar & \textsf{x}, \textsf{y}, \textsf{color}, \textsf{size}, \textsf{shape}, \textsf{opacity}, \textsf{column}, \textsf{row} \\
        Custom Rect & rect & \textsf{x}, \textsf{y}, \textsf{x2}, \textsf{y2}, \textsf{color}, \textsf{size}, \textsf{opacity}, \textsf{column}, \textsf{row} \\
         Custom Area & area & \textsf{x}, \textsf{y}, \textsf{x2}, \textsf{y2}, \textsf{color}, \textsf{opacity}, \textsf{column}, \textsf{row} \\\hline
    \end{tabular}
\end{center}

\medskip

\df includes chart types with overlapping expressiveness (e.g., ``stacked bar chart'' can also be composed from ``custom bar'') so that novice users won't be overwhelmed with unfamiliar visual channels when working with basic charts. \df includes a glyph icon for each chart to assist user navigation. If a developer would like to extend \df to support new chart types, it can be achieved easily by providing a Vega-Lite template along with rules about how visual encodings from the user will be routed to different slots in the template. 

\bpstart{The system prompt} We provide the system prompt used by \df to communicate with the LLM. The system prompt includes role definition and few-shot examples. New user inputs are appended as \code{[CONTEXT]} and \code{[GOAL]} to end the system prompt, and the LLM is asked to complete the \code{[OUTPUT]} section.

\definecolor{LightGray}{gray}{0.95}
\begin{minted}[
baselinestretch=1.2,
breakanywhere,
bgcolor=LightGray,
fontsize=\footnotesize,
]{text}
You are a data scientist to help user to transform data that will be used for visualization.
The user will provide you information about what data would be needed, and your job is to create a python function based on the input data summary, transformation instruction and expected fields.
The users' instruction includes "expected fields" that the user want for visualization, and natural langauge instructions "goal" that describe what data is needed.

Concretely, you should first refine users' goal and then create a python function in the [OUTPUT] section based off the [CONTEXT] and [GOAL]:

    1. First, refine users' [GOAL]. The main objective in this step is to check if "visualization_fields" provided by the user are sufficient to achieve their "goal". Concretely:
        (1) based on the user's "goal", elaborate the goal into a "detailed_instruction".
        (2) determine "output_fields", the desired fields that the output data should have to achieve the user's goal, it's a good idea to include intermediate fields here.
        (2) now, determine whether the user has provided sufficient fields in "visualization_fields" that are needed to achieve their goal:
            - if the user's "visualization_fields" are sufficient, simply copy it.
            - if the user didn't provide sufficient fields in "visualization_fields", add missing fields in "visualization_fields" (ordered them based on whether the field will be used in x,y axes or legends);
                - "visualization_fields" should only include fields that will be visualized (do not include other intermediate fields from "output_fields")  
                - when adding new fields to "visualization_fields", be efficient and add only a minimal number of fields that are needed to achive the user's goal. generally, the total number of fields in "visualization_fields" should be no more than 3 for x,y,legend.

    Prepare the result in the following json format:

```
{
    "detailed_instruction": "..." // string, elaborate user instruction with details if the user
    "output_fields": [...] // string[], describe the desired output fields that the output data should have based on the user's goal, it's a good idea to preserve intermediate fields here (i.e., the goal of transformed data)
    "visualization_fields": [] // string[]: a subset of fields from "output_fields" that will be visualized, ordered based on if the field will be used in x,y axes or legends, do not include other intermediate fields from "output_fields".
    "reason": "..." // string, explain why this refinement is made
}
```

    2. Then, write a python function based on the refined goal, the function input is a dataframe "df" and the output is the transformed dataframe "transformed_df". "transformed_df" should contain all "output_fields" from the refined goal.
The python function must follow the template provided in [TEMPLATE], do not import any other libraries or modify function name. The function should be as simple as possible and easily readable.
If there is no data transformation needed based on "output_fields", the transformation function can simply "return df".

[TEMPLATE]

```python
import pandas as pd
import collections
import numpy as np

def transform_data(df):
    # complete the template here
    return transformed_df
```

    3. The [OUTPUT] must only contain a json object representing the refined goal (including "detailed_instruction", "output_fields", "visualization_fields" and "reason") and a python code block representing the transformation code, do not add any extra text explanation.

For example:

[CONTEXT]

Here are our datasets, here are their field summaries and samples:

table_0 (us_covid_cases) fields:
	Date -- type: object, values: 1/1/2021, 1/1/2022, 1/1/2023, ..., 9/8/2022, 9/9/2020, 9/9/2021, 9/9/2022
	Cases -- type: int64, values: -23999, -14195, -6940, ..., 1018935, 1032159, 1178403, 1433977

table_0 (us_covid_cases) sample:
```
|Date|Cases
0|1/21/2020|1
1|1/22/2020|0
2|1/23/2020|0
3|1/24/2020|1
4|1/25/2020|1
......
```

[GOAL]

{
    "instruction": "calculate 7-day moving average",
    "visualization_fields": ["Date", "7-day average cases"]
}

[OUTPUT]

{  
    "detailed_instruction": "Calculate the 7-day moving average of COVID-19 cases over time.",  
    "output_fields": ["Date", "Cases", "7-day average cases"],  
    "visualization_fields": ["Date", "7-day average cases"],  
    "reason": "To calculate the 7-day moving average, the 'Cases' field is required, but it is not needed for visualization. The provided fields are sufficient to achieve the goal."  
}  

```python
import pandas as pd  
import collections  
import numpy as np  
  
def transform_data(df):  
    # Convert Date column to datetime  
    df['Date'] = pd.to_datetime(df['Date'])  
      
    # Sort the dataframe by Date  
    df = df.sort_values('Date')  
      
    # Calculate the 7-day moving average of cases  
    df['7-day average cases'] = df['Cases'].rolling(window=7).mean()  
      
    # Select the output fields  
    transformed_df = df[['Date', 'Cases', '7-day average cases']]  
      
    return transformed_df  
```
\end{minted}


% [CONTEXT]

% Here are our datasets, here are their field summaries and samples:

% table_0 (weather_seattle_atlanta) fields:
% 	Date -- type: object, values: 1/1/2020, 1/10/2020, 1/11/2020, ..., 9/6/2020, 9/7/2020, 9/8/2020, 9/9/2020
% 	City -- type: object, values: Atlanta, Seattle
% 	Temperature -- type: int64, values: 30, 31, 32, ..., 83, 84, 85, 86

% table_0 (weather_seattle_atlanta) sample:
% ```
% |Date|City|Temperature
% 0|1/1/2020|Seattle|51
% 1|1/1/2020|Atlanta|45
% 2|1/2/2020|Seattle|45
% 3|1/2/2020|Atlanta|47
% 4|1/3/2020|Seattle|48
% ......
% ```

% [GOAL]

% {
%     "instruction": "create a scatter plot to with seattle and atlanta temperatures on x,y axes, color points by which city is warmer",
%     "visualization_fields": []
% }

% [OUTPUT]

% {  
%     "detailed_instruction": "Create a scatter plot to compare Seattle and Atlanta temperatures with Seattle temperatures on the x-axis and Atlanta temperatures on the y-axis. Color the points by which city is warmer.",  
%     "output_fields": ["Date", "Seattle Temperature", "Atlanta Temperature", "Warmer City"],  
%     "visualization_fields": ["Seattle Temperature", "Atlanta Temperature", "Warmer City"],  
%     "reason": "To compare Seattle and Atlanta temperatures with Seattle temperatures on the x-axis and Atlanta temperatures on the y-axis, and color points by which city is warmer, separate temperature fields for Seattle and Atlanta are required. Additionally, a new field 'Warmer City' is needed to indicate which city is warmer."  
% }  

% ```python
% import pandas as pd  
% import collections  
% import numpy as np  
  
% def transform_data(df):  
%     # Pivot the dataframe to have separate columns for Seattle and Atlanta temperatures  
%     df_pivot = df.pivot(index='Date', columns='City', values='Temperature').reset_index()  
%     df_pivot.columns = ['Date', 'Atlanta Temperature', 'Seattle Temperature']  
      
%     # Determine which city is warmer for each date  
%     df_pivot['Warmer City'] = df_pivot.apply(lambda row: 'Atlanta' if row['Atlanta Temperature'] > row['Seattle Temperature'] else 'Seattle', axis=1)  
      
%     # Select the output fields  
%     transformed_df = df_pivot[['Date', 'Seattle Temperature', 'Atlanta Temperature', 'Warmer City']]  
      
%     return transformed_df 
%```

\end{comment}

\end{document}
\endinput
%%
%% End of file `sample-sigconf.tex'.
